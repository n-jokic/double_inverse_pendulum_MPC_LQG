\documentclass[a4paper,11pt]{article}
\usepackage{times}
%\setcounter{secnumdepth}{0} % sections are not getting numbered
\usepackage[english,serbian]{babel}


\usepackage[T1]{fontenc} 
\usepackage{biblatex} % bibliography

\addbibresource{citation.bib} % file with references


\usepackage{comment}
\usepackage{amsfonts}
\usepackage{amsmath}
\usepackage{amsthm}
\usepackage{IEEEtrantools}
\usepackage{graphicx}


%\usepackage{geometry}
%\usepackage{upgreek}
\usepackage[serbian]{babel}
%\usepackage{ulem}
%\usepackage{environ}
\usepackage{tikz}
\usepackage{color}
\usepackage{fancybox}

%\numberwithin{equation}{section}
\theoremstyle{definition} \newtheorem{deff}{Definicija}[section]
\theoremstyle{definition} \newtheorem{prim}[deff]{Primer}
\theoremstyle{plain} \newtheorem{teor}[deff]{Teorema}


\newcommand{\unija}[2]{#1 \cup #2}
\newcommand{\pres}[2]{#1 \cap #2}
\newcommand{\tnorm}{$t$-norm}
\newcommand{\tkonorm}{$t$-konorm}


\renewenvironment{proof}[1][\proofname]{{\bfseries #1.}}

\frenchspacing


\usepackage[a4paper,top=3cm,bottom=2cm,left=2cm,right=2cm,marginparwidth=1.75cm]{geometry}
%% Useful packages
\usepackage{mathrsfs}
\newsavebox\foobox
\newlength{\foodim}
\newcommand{\slantbox}[2][0]{\mbox{%
		\sbox{\foobox}{#2}%
		\foodim=#1\wd\foobox
		\hskip \wd\foobox
		\hskip -0.5\foodim
		\pdfsave
		\pdfsetmatrix{1 0 #1 1}%
		\llap{\usebox{\foobox}}%
		\pdfrestore
		\hskip 0.5\foodim
}}
\def\Laplace{\slantbox[-.45]{$\mathscr{L}$}}

%\usepackage[colorinlistoftodos]{todonotes}

\usepackage{caption}
\usepackage{subcaption}
\usepackage{changepage}


\usepackage{blindtext}

\usepackage{tabularx}
\usepackage[export]{adjustbox}

\usepackage[utf8]{inputenc}
\usepackage[T1]{fontenc}
\usepackage{lmodern}
\usepackage{graphicx}
\usepackage{color}
\usepackage{listings}
\usepackage{amsmath}

%\usepackage[usenames,dvipsnames]{xcolor}
%\usepackage[colorlinks=true,linkcolor=blue]{hyperref}

\usepackage{amsfonts}
\usepackage{epstopdf}

\usepackage{float}

\usepackage[shortlabels]{enumitem}
\usepackage[yyyymmdd]{datetime}


\renewcommand{\figurename}{Slika}

\DeclareMathOperator*{\argmax}{\arg\max}
\graphicspath{{./images/}}



\usepackage{siunitx}

\usepackage{scalerel}



\sloppy

\epstopdfsetup{update} % only regenerate pdf files when eps file is newer

%%%%%%%% DOCUMENT %%%%%%%%
\setlength {\marginparwidth }{2cm}
\begin{document}
	
	%%%% Title Page
	\begin{titlepage}
		
		\newcommand{\HRule}{\rule{\linewidth}{0.5mm}} 							% horizontal line and its thickness
		\center 
		
		% University
		\textsc{\LARGE Elektrotehnički fakultet u Beogradu}\\[1cm]
		
		% Document info
		\textsc{\Large Multivarijabilni sistema upravljanja}\\[0.2cm]
		\textsc{\large 13E054MSU}\\[1cm] 										
		\HRule \\[0.8cm]
		{ \huge \bfseries Reaktor sa kontinuiranim mešanjem (CSRT)}\\[0.7cm]								% Assignment
		%\HRule \\[2cm]
		\textsc{\large Sistem broj 5}\\[1cm]
		
		
		\large
		\vfill 
		\emph{Studenti:}\\
		Nikita Jokić 0139/2017\\[0.1cm]
		Ivona Dučić 0196/2019\\[1.5cm]		
		\emph{Mentori:}\\
		prof. dr Aleksandar Rakić\\[0.1cm]
		doc. dr Sanja Vujnović\\[1.5cm]										
		{\large Maj 2023}\\[2cm]
	\end{titlepage}
	\tableofcontents
	\newpage
	
	\section{Modeliranje sistema i analiza modela}
	\subsection{Uvod} 
	
	CSTR (Continuous Stirred Tank Reactor) sistem se često koristi za održavanje pH vrednosti i koncentracije kiseline u hemijskim procesima. U ovom sistemu, pH vrednost se održava pomoću kontrole protoka kiseline i baze. U velikom broju primena jedna od ovih veli"cina se odr"zava konstanom, a druga se koristi za finu korekciju pH vrednosti. Sistem za održavanje pH vrednosti i koncentracije kiseline može se primeniti u različitim industrijskim, laboratorijskim i hemijskim procesima gde je kontrola pH vrednosti važna. Neki od primera primene ovakvog sistema su:\\\\
	
	1. Hemijska i farmaceutska industrija: U proizvodnji hemikalija i farmaceutskih proizvoda, održavanje odgovarajuće pH vrednosti je ključno za postizanje željenih hemijskih reakcija, stabilnost proizvoda i kvalitet. Ovaj sistem može se koristiti za održavanje pH vrednosti tokom različitih faza proizvodnog procesa. \\
	
	2. Vodna postrojenja: U postrojenjima za prečišćavanje vode, održavanje optimalne pH vrednosti je važno za efikasno uklanjanje nečistoća, dezinfekciju i očuvanje kvaliteta vode. CSTR sistem može se koristiti za automatsko doziranje kiseline ili baze kako bi se održao pH unutar određenih granica. \\
	
	3. Laboratorijska istraživanja: U laboratorijskim uslovima, ovakav sistem može se koristiti za održavanje stabilne pH vrednosti tokom eksperimenata ili testiranja. To je posebno važno kada se radi sa reaktivnim supstancama ili kada je precizna kontrola pH neophodna. \\
	
	4. Hrana i piće: U prehrambenoj industriji, održavanje odgovarajuće pH vrednosti može biti važno za konzervaciju hrane, fermentaciju i kontrolu ukusa. CSTR sistem može se koristiti za održavanje pH vrednosti tokom proizvodnog procesa hrane i pića. \\
	
	\begin{figure}[!h]
		\centering
		\includegraphics[width=0.4\linewidth]{slike/zad1/cstr_slika.png}
		\caption{Reaktor sa kontinuiranim mešanjem}
		\label{fig:cstr}
	\end{figure}
	
	Kao "sto vidimo na slici (Sl. \ref{fig:cstr}) glavni elementi CSTR sistema uključuju reaktor, koji je obično veliki rezervoar sa mešalicom koja kontinuirano meša reaktante, kao i dodatnom opremom za kontrolu protoka kiseline i baze. Ovi protoci se prilagođavaju kako bi se održala željena pH vrednost i koncentracija kiseline. \\
	
	Kada pH vrednost u reaktoru padne ispod željenog nivoa, sistem dodaje bazu (npr. natrijum hidroksid) u reaktor. Protok baze se povećava kako bi se neutralisao višak kiseline i podigla pH vrednost. S druge strane, kada pH vrednost postane previsoka, sistem dodaje kiselinu (npr. sumpornu kiselinu) kako bi se snizila pH vrednost. Protok kiseline se povećava u ovom slučaju. Mo"zemo primetiti da je u pitanju proces sa negativnim stati"ckim poja"canjem, u slu"caju da se upravlja protokom kiseline.
	
	
	
	
	
	\newpage
	
	\subsection{Modeliranje sistema} 
	
	Kao što smo već spomenuli, zadatak CSTR je održavanje konstante pH vrednosti tečnosti ($\mathrm{y}$) i koncentracije kiseline u rezervoaru ($\mathrm{x_a}$), adekvatnom korekcijom protoka kiseline ($\mathrm{F_a}$) i baze ($\mathrm{F_b}$) koji se dodaju u rezervoar. Upravljanje protokom kiseline i baze se obično vrši pomoću regulatora koji kontrolišu ventile koji regulišu protok supstanci. Regulatori dobijaju povratne informacije o pH vrednosti iz senzora smeštenih u reaktoru. Pomoću tih informacija, regulatori automatski prilagođavaju protok kiseline i baze kako bi održali pH vrednost na željenom nivou.\\
	
	Kao meru kiselosti ili alkalnosti rastvora uvodi se pH vrednost, pH se defini"se kao negativni logaritam koncentracije jonizovanih vodoničnih jona ($H^+$). Dalje u tekstu se kao simbol za koncentraciju jona koristi $[X]$.
	
	\begin{equation}
		\text{pH} = -\text{log}_{10}([H^+])
	\end{equation}
	
	\noindent Veza između pH vrednosti i koncentracije kiseline je obrnuta proporcionalna.  Kada koncentracija kiseline raste, broj jonizovanih vodoničnih jona ($H^+$) takođe raste, što dovodi do sniženja pH vrednosti. To znači da veće koncentracije kiseline imaju niže pH vrednosti, što ukazuje na veću kiselost rastvora.\\
	
	U literaturi se model izvodi pod pretpostavkom da je zapremina sistema $V$ konstantna, dolazi do savr"senog me"sanja i proces je izotermi"cki. Bez gubitka op"stosti mo"zemo pretpostaviti da se u procesu koristi hlorovodoni"cna kiselina HCl, a kao baza se koristi natrijum-hidroksid NaOH. Iz hemije je poznato da kiseline i baze ne podle"zu potpunoj disocijaciji u vodi, ve\'c dolazi do nastanka dinami"ckog ekvilibrijuma.
	Me\dj{}utim kako su HCl i NaOH jaka kiselina/baza mo"zemo smatrati da je disocijacija potpuna. Dalje usvajamo oznake $x_a = [\text{Cl}^-]$ i $x_b = [\text{Na}^+]$. Promenu koncentracija $\dot{x}_a$ i $\dot{x}_b$ mo"zemo modelovati kao spregnuti NF proces 1. reda. Odnosno: 
	\begin{align}\label{eq:ion_balance}
		&V\dot{x}_a = F_aC_a - (F_a + F_b)x_a \\
		&V\dot{x}_b = F_bC_b - (F_a + F_b + k_v)x_b 
	\end{align}
	"Clanovi $F_aC_a$ i $F_bC_b$ predstavljaju dotok novih jona u rezervoar, "clanovi $-(F_a + F_b)x_a$ i $-(F_a + F_b)x_b$ su posledica konstatne zapremine rezervoara, a "clan $-k_vx_b$ opisuje konstatni pad koncentracije baze usled reakcije sa proizvodima u reaktoru. Kona"cno nam preostaje da odredimo pH vrednost rastvora, to je najlak"se u"ciniti indirektno koriste\'ci zakon odr"zanja naelektrisanja, odnosno mora da va"zi slede\'ci izraz: 
	\begin{equation}\label{eq:charge_balance}
		x_a + [\text{OH}^-] = x_b + [\text{H}^+]
	\end{equation}
	Nepoznatu $[\text{OH}^-]$ mo"zemo da elimini"semo primenom jedna"cine za konstantu disocijacije vode $k_w$.
	
	\begin{equation}\label{eq:k_w}
		[\text{OH}^-][\text{H}^+] = k_w
	\end{equation}
	Kombinacijom jedna"cina \eqref{eq:charge_balance} i \eqref{eq:k_w} dobijamo slede\'cu kvadratnu jedna"cinu:
	\begin{equation}\label{eq:pH_square}
		[\text{H}^+]^2 + (x_a - x_b)[\text{H}^+] - k_w = 0
	\end{equation} 
	Kona"cno re"savanjem kvadratne jedna"cine \eqref{eq:pH_square} dobijamo izraz za pH vrednost rastvora koji glasi: 
	\begin{equation}\label{eq:pH}
		y = -\log\left(\sqrt{\frac{(x_b - x_a)^2}{4} + kw} - \frac{x_b - x_a}{2}\right)
	\end{equation}
	
	\clearpage
	
	
	
	Kombinacijom jedna"cina \eqref{eq:ion_balance} i \eqref{eq:pH}, dobijamo nelinearan dinamički model sistema, koji glasi : 
	\begin{equation}\label{eq: model u prostoru}
		\begin{split}
			&V\dot{x}_a(t) = F_aC_a - (F_a + F_b(t))x_a(t) \\
			&V\dot{x}_b(t) = F_b(t)C_b - (F_a + F_b(t) + kv)x_b(t) \\
			&y(t) = -\log\left(\sqrt{\frac{x_b(t) - x_a(t)}{4} + kw} - \frac{x_b(t) - x_a(t)}{2}\right)
		\end{split}
	\end{equation}
	\vspace{0.5cm}
	
	Prva jednačina opisuje promenu koncentracije komponente a ($\mathrm{x_{a}}$) tokom vremena (t). Promena koncentracije je rezultat ulaznog protoka kiseline ($\mathrm{F_{a}C_{a}}$) koji ulazi u reaktor i izlaznog protoka kiseline ($\mathrm{(F_{a}+F_{b})x_{a}}$) koji napušta reaktor. Ova jednačina predstavlja zakon očuvanja mase za komponentu a.\\
	
	Druga jednačina opisuje promenu koncentracije komponente b ($\mathrm{x_{b}}$) tokom vremena (t). Promena koncentracije je rezultat ulaznog protoka baze ($\mathrm{F_{b}C_{b}}$) koji ulazi u reaktor, izlaznog protoka kiseline ($\mathrm{(F_{a}+F_{b})x_{b}}$) koji napušta reaktor i reakcije između baze i proizvoda u reaktoru, koja se odvija sa brzinom konstante ($\mathrm{k_{v}}$). Ova jednačina takođe predstavlja zakon očuvanja mase za komponentu b. \\
	
	Treća jednačina izračunava pH vrednost (y) na osnovu koncentracija kiselih i baznih komponenti ($\mathrm{x_{a}}$ i $\mathrm{x_{b}}$). Ova jednačina koristi logaritamsku funkciju kako bi se izračunala pH vrednost na osnovu razlike između kiselih i baznih komponenti. \\
	
	Konstanta $\mathrm{k_{w}}$ predstavlja konstantu disocijacije vode.  To je matematička konstanta koja opisuje dinami"cku ravnotežu između koncentracije vodoni"cnih-jona [$\text{H}^+$] i hidroksid-jona [$\text{OH}^-$] u vodenom rastvoru. Konstanta disocijacije vode ima vrednost od približno $\mathrm{10^{-14} mol{^2}/L{^2}}$ pri standardnim uslovima (25°C). Ova konstanta se može koristiti za izračunavanje koncentracija vodoni"cnih-jona i hidroksid-jona u vodenim rastvorima. \\
	
	Nominalne vrednosti parametara sistema su : $\mathrm{C_{a} = {C_{b}} = 10^{-14} \: mol{^2}/l{^2}, k_{w} = 10^{-14} \: mol{^2}/l{^2}}$, \\ $\mathrm{k_{v} = 0.01 \: l/s  }$. Ne mo"zemo mnogo komentarisati o uobi"cajenim vrednostima ovih parametara zato "sto su primene CSTR reaktora previ"se raznolike u praksi.\\\\
	
	
	\noindent Osim navedenog modela u literaturi se mo"ze prona\'ci jo"s nekoliko pristupa modelovanju:\\
	
	\textbf{Neutalizacione krive}:
	
	\noindent Jedan od originalnih pristupa je podrazumevao ekperimentalno odre\dj{}ivanje neutalizacionih krivih $f_{\text{pH}}(x_b)$, me\dj{}utim problem kod ovog pristupa je "sto te krive inherentno opisuju samo steady-state pona"sanje sistema, pa se ne mogu korisiti za efikasno ubla"zavanje tranzijenta. Pored problematike nedostatka opisa dinamike sistema u praksi se "cesto koriste kombinacije slabih kiselina i jakih baza, u tom slu"caju dolaze do nastanka izra"zenih nelinearnosti netralizacionih krivih zbog formiranja pufera. Usled prisustva dinamike pufera proces se efektivno kre\'ce izme\dj{}u nekoliko neutralizacionih krivih, tip krive zavisi od koncentracije pufera. Ovaj pristup se mo"ze koristiti ako neutralizaciona kriva ne sadr"zi foldback ta"cke, naime u odre\dj{}enim situacijama neutralizaciona kriva prolazi kroz lokalni minimum, pa i smanjenje i pove\'canje protoka baze dovodi do pove\'canja pH vrednosti, "sto nije prihavtivo sa stanovni"stva klai"cnih zakona upravljanja.\\
	\clearpage
	\textbf{Modelovanje dinamike pufera}:\\
	\noindent Slede\'ci pristup poku"sava da modeluje interakcije slabe kiseline i jake baze koriste\'ci slede\'ce jedna"cine: 
	\begin{align}\label{eq:pufer}
		&[\text{AC}^-][\text{H}^+] = K_a [\text{HAC}]\\
		&[\text{Na}^+][\text{H}^+] = [\text{AC}^-][\text{OH}^-]
	\end{align}
	Kombinuju\'ci jedna"cine \eqref{eq:k_w} i \eqref{eq:pufer} dobijamo slede\'ci izraz za pH vrednost:
	\begin{equation}\label{eq:pH_3rd_deg}
		[\text{H}^+]^3 + [\text{H}^+]^2(K_a + x_b) + [\text{H}^+](K_a(x_b - x_a) -k_w) - k_ak_w = 0
	\end{equation}
	U praksi se pokazalo da \eqref{eq:pH_3rd_deg} dobro modeluje procese kod kojih je izra"zena dinamika pufera, osim toga mo"ze se primetiti da se za jake kiseline (tj. $K_a$ je dovoljno veliko) model svodi na \eqref{eq:pH_square}. Me\dj{}utim ni ovaj model nije savr"sen i ne opisuje dinamiku procesa u potpunosti, poku"saji da se ovaj model unapredi se svode na modelovanje dinamike promene pH vrednosti.\\
	
	\textbf{Modelovanje dinamike promene pH vrednosit}:\\
	\noindent Prvi pristup modelovanja dinamike promene pH vrednosti poku"sava da opi"se NF prirodu pH vrednosti u velikim rezervoarima. U model se uvodi nova promenljiva stanja koja se modeluje kao:
	\begin{equation}\label{eq:nf_pH}
		T[\dot{\text{H}}^+] = G - [\text{H}^+] 
	\end{equation}
	U jedna"cini \eqref{eq:nf_pH} T predstavlja $T = \frac{V}{F_a + F_b}$, a G je nepoznata funkcija. Obi"cno se uvodi pretpostavka da je G sporo promenljiva funckja, pa je mogu\'ce izvr"siti adaptaciju i estimaciju parametara iste. Nekad se "cak usvaja pretpostavka da je G realna konstanta. Za ve\'cinu upotreba ovaj model se pokazao kao dovoljno jednostavan za implementaciju, ali i dovoljno precizan za potrebe upravljanja. Drugi pristup podrazumeva analiti"cko izvo\dj{}enje izraza za $[\dot{\text{H}}^+]$ na osnovu \eqref{eq:pH_3rd_deg}. Dobija se slede\'ci izraz:
	\begin{equation}\label{eq:pH_state}
		V[\dot{\text{H}}^+] = \frac{-[\text{H}^+]^2(F_bCb -(F_a+F_b)x_b) - K_a[\text{H}^+](F_bC_b-F_aC_a-(F_a+F_b)(x_b-x_a))}{3[\text{H}^+]^2 + 2[\text{H}^+](K_a + x_b) + K_a(x_b-x_a) - k_w}
	\end{equation}
	Pokazuje se da jedna"cina \eqref{eq:pH_state} skoro savr"seno opisuje dinamiku promene pH vrednosti, pa se "cesto koriti prilikom implementacije nelinearnih zakona upravljanja ili zakona upravljanja na bazi predikcije (MPC).\\\\\\
	
	
	
	
	
	
	
	
	Sledeće što je potrebno uraditi jeste izbor nominalnog režima. Naime, nominalni režim sistema dobijamo kada prve izvode stanja sistema izjednačimo sa nulom (u našem slučaju to je $\mathrm{\dot{x}_a = 0, \dot{x}_b = 0 }$). Zamenom u jednačine dimaničkog modela, dobijamo sistem od tri jednačine i pet nepoznatih. Samim tim, očigledno je da ovakav sistem ima beskonačno puno rešenja. Kako bismo rešili sistem, usvojićemo vrednosti za dve nepoznate. Najlogičnije je izabrati nominalni režim za veličine koje kontrolišemo, odnosno pH vrednost i količina koncentracije kiseline u rezervoaru, odnosno y i $\mathrm{x_a}$, a na osnovu njih ćemo izračunati upravljanje koje dovodi sistem u nominalni režim. Obzirom da CSTR sistem ima različite primene, teško je definisati uzak opseg za date promenljive. Na kraju smo se odlučili da će zadatak CSTR sistema biti da drži supstancu u reaktoru neutralnom, odnosno da pH vrednost iznosi 7. Razmotrivši literaturu dostupnu na internetu, možemo izabrati vrednost za koncentraciju kiseline u reaktoru. U neutralnom stanju, pH vrednost 7 ukazuje na ravnotežu između [$\text{H}^+$] i [$\text{OH}^-$] jona u rastvoru. Uobičajeno se uzima da je koncentracija kiseline ($\mathrm{x_a}$) jednaka koncentraciji baze ($\mathrm{x_b}$) kako bi se održala neutralnost. Ove vrednosti su obično na nivou od $\mathrm{10^{-7}}$do $\mathrm{10^{-6}}$ mol/l. Na osnovu svega, izabrali smo da je nominalna vrednost za koncentraciju kiseline u reaktoru $\mathrm{x_a = 3.125~10^{-7} \: mol/l }$. Sada imamo sistem od tri jednačine sa tri nepoznate, čime je dobijamo jedinstveno rešenje, $\mathrm{x_b = 3.125~10^{-7} \: mol/l}, \mathrm{F_a = 0.0026 \: l/s}, \mathrm{F_b = 0.0057 \: l/s }$ .\\
	\clearpage
	
	
	
	\subsubsection{Simulink model}
	
	Simulink model (Sl. \ref{fig:model}) predstavlja realizaciju modela u prostoru stanja (\ref{eq: model u prostoru}). Sadrži 2 ulaza i 3 izlaza:
	\begin{itemize}
		\item Ulaz 1 - protok $F_a$
		\item Ulaz 2 - protok $F_b$
		\item Izlaz 1 - stanje $x_a$
		\item Izlaz 2 - merenje $y$
		\item Izlaz 3 - stanje $x_b$
	\end{itemize}
	\vspace{2cm}
	
	\begin{figure}[!h]
		\centering
		\includegraphics[width=1\linewidth]{slike/zad1/neli_model.png}
		\caption{Dinami"cki model CSTR reaktora}
		\label{fig:model}
	\end{figure}
	\vspace{2cm}
	
	U modelu su kori"s\'ceni slede\'ci parametri: $k_v$ = 0.01 l/s, $V$ = 30 l, $k_w$ = $10^{-14}$ $\text{mol}^2$/$\text{l}^2$, $C_a$ = $10^{-14}$ mol/l, $C_b$ = $10^{-14}$ mol/l. A kao po"cetna stanja za integratore je usvojeno $x_{a0}$ = 0 mol/l i $x_{b0}$ = 0 mol/l.
	
	
	
	
	
	\clearpage
	
	
	\subsection{Ponašanje sistema u otvorenoj sprezi}
	
	Simulink model (Sl. \ref{fig:ol_model}) je kori"s\'cen prilikom ispitivanja pona"sanja sistema u otvorenoj sprezi.
	
	\begin{figure}[!h]
		\centering
		\includegraphics[width=0.9\linewidth]{slike/zad1/open_loop.png}
		\caption{open loop model}
		\label{fig:ol_model}
	\end{figure}
	
	
	Prelazak sistema iz po"cetnog stanja ($x_a = x_b = 0$) ka nominalnom re"zimu rada je dat na slici (Sl. \ref{fig:ol_tran}). Vreme smirenja, za promenljive stanja $x_a$ i $x_b$, iznosi 4010 s, odnosno 1830 s, a za mereni izlaz $y$ 34890 s.
	
	\begin{figure}[!h]
		\centering
		\includegraphics[width=0.7\linewidth]{slike/zad1/Transient.eps}
		\caption{Tranzijent sistema ka nominalnom re"zimu rada }
		\label{fig:ol_tran}
	\end{figure}
	
	
	
	\clearpage 
	\subsection{Linearizacija sistema}
	
	Nakon što smo izabrali nominalne vrednosti parametara sistema, moguće je izvršiti Jakobijan linearizaciju u blizini nominalnog režima. Linearizacija se vrši isto kao i u slučaju SISO sitema :
	
	\begin{equation}
		\begin{split}
			\Delta \dot{\textbf{x}} = \textbf{A}\Delta \textbf{x} &+ \textbf{B}\Delta \mathbf{u} \\
			\Delta y = \textbf{C}\Delta \textbf{x} &+ \textbf{D}\Delta \mathbf{u} \\
			\textbf{A} =\frac{\partial  \textbf{f}(\textbf{x},\mathbf{u})}{\partial \textbf{x}}|_{\textbf{x}=\mathbf{x_e} ~ \mathbf{u}=\mathbf{u_e}}& ~~~~~ \textbf{B} =\frac{\partial  \textbf{f}(\textbf{x},\mathbf{u})}{\partial \mathbf{u}}|_{\textbf{x}=\mathbf{x_e} ~ \mathbf{u}=\mathbf{u_e}} \\
			\textbf{C} =\frac{\partial  \mathbf{h}(\textbf{x},\mathbf{u})}{\partial \textbf{x}}|_{\textbf{x}=\mathbf{x_e} ~ \mathbf{u}=\mathbf{u_e}}&  ~~~~~ \textbf{D} = \frac{\partial  \mathbf{h}(\textbf{x},\mathbf{u})}{\partial \mathbf{u}}|_{\textbf{x}=\mathbf{x_e} ~ \mathbf{u}=\mathbf{u_e}}
		\end{split}
	\end{equation}
	
	Primenom simboličkog tool-boxa dobijamo sledeće izraze za matrice \textbf{A}, \textbf{B}, \textbf{C} i \textbf{D}, i funkciju prenosa linearizovanog modela $G(s)$: 
	
	
	
	
	\begin{equation}
		\begin{split}
			\textbf{A} &= 10^{-3}\begin{bmatrix}
				-0.2778 & 0 \\
				0 & -0.6111
			\end{bmatrix} 			
			~~~~~ \textbf{B} =10^{-7}\begin{bmatrix}
				0.2292 &  -0.1042\\
				-0.1042 &   0.2292
			\end{bmatrix}  \\						
			\textbf{C} &= 10^{6}\begin{bmatrix} 
				0 & 0\\
				-2.1715 &   2.1715
			\end{bmatrix}   ~~~~~~~~~~~~~~~~~ \textbf{D} = 0
		\end{split}
	\end{equation}
	
	\begin{equation}
		\begin{split}
			G(s) = \textbf{C}(s\textbf{I}-\textbf{A})^{-1}\textbf{B} + \textbf{D} = 
			\begin{bmatrix}
				\frac{2.292~10^{-6}}{s+ 0.0002778} &  \frac{-0.07238s - 3.669~10^{-5}}{s^2 + 0.000889s + 1.698~10^{-7}} \\
				\frac{-1.042~10^{-8}}{s + 0.0002778} & \frac{0.0723s + 2.765~10^{-5}}{s^2 + 0.000889s + 1.698~10^{-7} }
			\end{bmatrix}
			\\
		\end{split}
	\end{equation}
	
	Na slici (Sl. \ref{fig:nel_lin}) prikazano je ponašanje linearizovanog sistema u odnosu na originalni nelinearni model. Naizmenične odskočne promene upravljačkih signala iznose 10\% nominalne vrednosti. Vidimo da se grafici poklapaju u okolini radne tačke, dok na krajevima odskočnih promena dolazi do male razlike. 
	
	\begin{figure}[!h]
		\centering
		\includegraphics[width=0.67\linewidth]{slike/zad1/Nelin_vs_lin.eps}
		\caption{Ponašanje originalnog i linearizovanog CSTR sistema }
		\label{fig:nel_lin}
	\end{figure}
	
	\clearpage
	\subsection{Poremećaji u sistemu}
	
	
	Poremećaji u opštem smislu predstavljaju neočekivane promene ili smetnje u normalnom funkcionisanju sistema. Oni mogu uzrokovati odstupanje regulisanih veličina od željenih vrednosti, što može dovesti do nepoželjnih efekata ili narušavanja procesa. Poremećaji u CSTR sistemu mogu biti uzrokovani različitim faktorima, kao što su :\\
	
	- Promene u protoku sirovina: Promene u protoku kiseline, baze ili drugih hemikalija koje se dodaju u sistem mogu dovesti do poremećaja. Povećanje ili smanjenje protoka može uticati na brzinu reakcija i ravnotežu sistema, što može rezultirati promenom pH vrednosti ili koncentracije kiseline.\\\\
	
	- Otkaz opreme: Otkaz senzora pH, pumpe ili drugih komponenti sistema može dovesti do poremećaja u kontroli pH vrednosti i koncentracije kiseline. Na primer, otkaz senzora pH može rezultirati netačnim merenjima i pogrešnim upravljanjem sistema. \\\\
	
	- Kontaminacija ili nečistoće: Prisustvo nečistoća, kontaminanata ili stranih materija u kiselini, bazi ili procesnom sistemu može poremetiti pH vrednost i koncentraciju kiseline. Kontaminacija može uticati na tačnost merenja i ometati pravilan rad senzora, dovodeći do poremećaja u kontroli pH vrednosti. \\\\
	
	- Neregularnosti u hemijskim reakcijama: Određene hemijske reakcije ili ravnoteže mogu biti osetljive na promene u temperaturi, pritisku, koncentraciji ili drugim faktorima. Poremećaji u tim parametrima mogu dovesti do odstupanja pH vrednosti i koncentracije kiseline od željenih vrednosti. \\\\\\
	
	Pomenuti poremećaji direktno ili indirektno deluju na promenu pH vrednosti. Povećanje protoka kiseline može dovesti do snižavanja pH vrednosti, dok povećanje protoka baze može dovesti do povećanja pH vrednosti. U slučaju nečistoća, ako se u sistem unese kiselinski ili bazni materijal koji nije dobro izbalansiran, može doći do promene pH vrednosti. Temperatura može imati uticaj na pH vrednost u hemijskim sistemima zbog promena u hemijskim ravnotežama i jonizaciji kiselina i baza. Samim tim, možemo zaključiti da je za sve ove izvore poreme\'caja  zajedni"cko da se ogledaju u promeni pH vrednosti, pa se prirodno name\'ce izbor za modelovanje poreme\'caja - poreme\'caj \'cemo modelovati kao aditivnu step promenu merene pH vrednosti. Kao maksimalnu vrednost poreme\'caja usvajamo $\pm 0.8 ~ \text{pH}$, razlog za usvajanje ove vrednosti je to "sto ovo predstavlja jako veliko odstupanje od "zeljene koncentracije $[\text{H}^+]$ jona (pH skala je logaritamska). \\
	
	Promena pH vrednosti ne mo"ze biti trenutna (osim ako nije do"slo do kvara na samom senzoru), pa nije lo"sa ideja propustiti step poreme\'caj kroz NF filtar kako bismo dobili poreme\'caj koji ima vi"se fizi"ckog smisla. U liteaturi se obi"cno ispita pona"sanje regulatora na step poreme\'caj, a zatim ponovi ispitivanje za filtriran poreme\'caj. Na ovaj na"cin mo"zemo videti pona"sanje kontrolera u najgorem slu"caju, ali i pona"sanje u realnoj situaciji.\\
	
	\noindent Odnosno koristi\'cemo slede\'ce modele poreme\'caja: 
	\begin{align}
		&G_{d1}(s) = 1 \\
		&G_{d2}(s) = \frac{1}{\frac{s}{w_p} + 1}
	\end{align}
	
	
	
	\clearpage
	\subsection{Upravljački signali i skaliranje signala}
	
	U literaturi se mo"ze prona\'ci informacija da se, efikasnosti radi, u industriji te"zi da se sve dimenzioni"se prema vrednostima iz nominalog re"zima rada, pa se kao gornja granica za upravljanja mo"ze usvojiti 120\% vrednosti iz nominalnog re"zima rada. Kao donju granicu usvajamo 0, zato "sto fizi"cki nije mogu\'ce da precizno elimini"semo jone iz reaktora.\\ 
	
	Kao "sto mo"zemo videti sa (Sl. \ref{fig:ol_tran}) stanja, upravljanja i mereni izlaz imaju neuporedive redove veli"cina, kao i jedinice - "sto mo"ze zakomplikovati proces projektovanja. Kako bismo od"zali uniformnost i u"cinli grafike preglednijim uve"s\'cemo slede\'ce matrice skaliranja:
	\begin{equation}
		\begin{split}
			&D_y = \text{diag}(x_{a0}, y_0)\\
			&D_u = \text{diag}(1.2F_{a0}, 1.2F_{b0})\\
			&D_d = y_0
		\end{split}
	\end{equation}
	Skaliranje mo"zemo izvr"siti kao $x = D^{-1}_x \hat{x}$. Jedino ostaje pitanje kako da preskalirano linearizovani model sistema, kao i rezultuju\'ci kontroler. To mo"zemo da postignemo na slede\'ci na"cin: 
	\begin{equation}\label{eq:skaliranje}
		\begin{split}
			&\hat{y} = \hat{G}\hat{u} + \hat{G}_d\hat{d}\\
			&D_yy = \hat{G}D_uu + \hat{G}_dD_dd\\
			&y = D^{-1}_y\hat{G}D_uu + D^{-1}_y\hat{G}_dD_dd \\
			&u = Kv\\
			&\hat{u} = D_uKD^{-1}_y\hat{v}
		\end{split}
	\end{equation}
	Na osnovu jedna"cina \eqref{eq:skaliranje} mo"zemo definisati skaliranu matricu funkcija prenosa objekta upravljanja:
	\begin{equation}
		G = D^{-1}_y\hat{G}D_u, 
	\end{equation} skaliranu matricu funkcija prenosa modela poreme\'caja: 
	\begin{equation}
		G_d = D^{-1}_y\hat{G}D_d , 
	\end{equation} kao i deskalirani kontroler: 
	
	\begin{equation}
		\hat{K} = D_uKD^{-1}_y
	\end{equation}
	
	
	\newpage
	\subsection{Multivarijabilne nule i polovi}
	
	Kao i u slučaju SISO sistema, i za MIMO sisteme, pokazatelji stabilnosti i performansi sistema su polovi i nule sistema. Za razliku od SISO sistema, gde su jasno uočljivi polovi i nule sistema, za MIMO sistem potrebno je izvršiti određenu analizu kako bismo odredili broj i položaj nula i polova sistema. \\
	
	\textbf{Definicija}: Polovi $\mathrm{p_i} = $ sistema, datog modela u prostoru stanja G = ss(A,B,C,D), su karakteristične vrednosti $\mathrm{\lambda_i(A)} $, i = 1,2,..n., matrice A. 
	Za karakteristični model:
	
	\begin{equation}
		f(s) = \det(s\mathbf{I}-A) = \prod_{i=1}^{n} (s - p_i)
	\end{equation}
	
	
	\noindent polovi sistema su nule polinoma f(s), tj. rešenja karakteristične j-ne f(s) = 0.
	Napomena: Ako model sistema nije minimalna realizacija (model nije minimalnog mogućeg reda), postoje polovi sistema koji odgovaraju nekontrolabilnim i/ili neopservabilnim stanjima.
	Ako je model sistema u formi matrice funkcija prenosa, za određivanje polova sistema, može se upotrebiti sledeća teorema. \\
	
	\textbf{Teorema:} Karakteristični polinom f(s), koji odgovara minimalnoj realizaciji matrice funkcija prenosa G(s), je najmanji zajednički imenilac svih nenultih minora svih redova matrice G(s).
	
	Posmatrajući matricu funkcije prenosa, možemo odmah reći na kojim pozicijama se nalaze polovi sistema, ali ne i broj polova. \\
	
	Na osnovu polova sistema, možemo komentarisati stabilnost sistema. Sistem, predstavljen modelom G(s) = ss(A, B, C, D) prostora stanja, je stabilan ako su sve karakteristične vrednosti $\mathrm{\lambda_i(A)} $, i = 1,2,..n, 
	matrice A u levoj poluravni s-ravni, tj. ($\forall i$)Re($\mathrm{\lambda_i(A)} $ )< 0. \\
	
	\textbf{Definicija:} (Kompleksna) vrednost s = $z_i$ je nula sistema G(s), ako je rang(G($z_i$)) manji od normalnog ranga G(s).\\
	
	Polinom nula se definiše kao: 
	
	\begin{equation}
		z(s) =  \prod_{i=1}^{n_z} (s - z_i)
	\end{equation}
	
	\noindent gde $n_z$ predstavlja broj konačnih nula G(s).
	Normalan rang G(s) je rang G(s) za sve vrednosti s, osim za konačan broj singularnosti (koje predstavljaju nule sistema).
	Napomena: Prethodna definicija nula podrazumeva minimalnu realizaciju G(s) sistema. Ove nule nazivaju se i “transmisionim nulama” ili “multivarijabilnim nulama”, da bi ih razlikovali od nula pojedinačnih funkcija prenosa unutar matrice funkcija prenosa.\\
	
	Obzirom da je naš sistem kvadratnog reda, možemo primeniti i sledeću teoremu: \\
	
	\textbf{Teorema:} Za kvadratni sistem, predstavljen matricom funkcija prenosa G(s), rešenja s = $z_i$ jednačine det(G(s)) = 0 su nule minimalne realizacije. \\
	
	Primenom ugrađenih MATLAB funkcija \textbf{tzero} i \textbf{tpole}, dobijamo da sistem nema nule i da ima dva pola u levoj poluravni i oni iznose: $p_1 = -6.1111~10^{-4}$ , $p_2 = -2.7778~10^{-4}$ .
	
	\newpage
	
	\subsection{Dijagonalna dominantnost}
	
	RGA (engl. Relative Gain Array – RGA) matrica predstavlja meru interakcije objekta upravljanja. Pretpostavićemo da je broj ulaza i izlaza sistema jednak, i neka je G(s) matrica funkcije prenosa, a H(s) inverzna funkcija prenosa, tj H(s) = $G(s)^{-1}$, te važe sledeće relacije: 
	\begin{equation}
		\begin{bmatrix}
			y_1 \\
			\vdots \\
			y_n
		\end{bmatrix}
		= 
		\begin{bmatrix}
			G_{11}(s) & \cdots & G_{1n}(s) \\
			\vdots & \ddots & \vdots \\
			G_{n1}(s) & \cdots & G_{nn}(s)
		\end{bmatrix}
		\begin{bmatrix}
			u_1 \\
			\vdots \\
			u_n
		\end{bmatrix}
		\label{eq:matricaG}
	\end{equation}
	
	\begin{equation}
		\begin{bmatrix}
			u_1 \\
			\vdots \\
			u_n
		\end{bmatrix}
		= 
		\begin{bmatrix}
			H_{11}(s) & \cdots & H_{1n}(s) \\
			\vdots & \ddots & \vdots \\
			H_{n1}(s) & \cdots & H_{nn}(s)
		\end{bmatrix}
		\begin{bmatrix}
			y_1 \\
			\vdots \\
			y_n
		\end{bmatrix}
		\label{eq:matricaH}
	\end{equation}
	
	Neka je zadatak da se upravljačkim signalom 
	$u_j$ kontroliše izlaz $y_i$. Razmotrimo dva ekstremna slučaja: \\
	
	Prvi slučaj kada su sve druge petlje regulacije otvorene $(\forall k \neq j) \, u_k = 0$. Tada se na osnovu jednačine 
	\eqref{eq:matricaG}, dobija pojačanje objekta upravljanja od posmatranog ulaza do posmatranog izlaza : 
	
	\begin{equation}
		\frac{{\partial y_i}}{{\partial u_j}} \bigg|_{(\forall k \neq j) \, u_k=0} = G_{ij}(s)
		\label{eq:pojac}
	\end{equation}
	
	Drugi slučaju kada su sve druge regulacije zatvorene perfektnim upravljanjem, tj $(\forall k \neq i) \, y_k = 0$. Tada se pojačanje dobija na osnovu jednačine \eqref{eq:matricaH}, i glasi: 
	
	\begin{equation}
		\frac{{\partial u_j}}{{\partial y_i}} \bigg|_{(\forall k \neq i) \, y_k=0} = H_{ji}(s)
	\end{equation}
	A na osnovu činjenice da je matrica H jednaka inverziji matrice G, možemo izraz \eqref{eq:pojac} napisati u drugačijem obliku :
	\begin{equation}
		\frac{{\partial y_i}}{{\partial u_j}} \bigg|_{(\forall k \neq i) \, y_k=0} = \frac{1}{{H_{ji}(s)}} = H_{ji}^{-1}(s)
	\end{equation}
	
	Zapravo, ono što se koristi kao mera relativnog pojačanja jeste relativno pojačanje i računa se kao:
	
	\begin{equation}
		\Lambda_{ij}(s) = \frac{{\frac{{\partial y_i}}{{\partial u_j}} \bigg|_{(\forall k \neq j) \, u_k=0}}}{{\frac{{\partial y_i}}{{\partial u_j}} \bigg|_{(\forall k \neq i) \, y_k=0}}} = G_{ij}(s) H_{ji}(s)
		\label{eq:pojac_lambda}
	\end{equation}
	
	RGA je matrica svih relativnih pojačanja objekta upravljanja:
	
	\begin{equation}
		RGA(G(s)) = \begin{bmatrix}
			\Lambda_{11}(s) & \Lambda_{12}(s) & \cdots & \Lambda_{1n}(s) \\
			\Lambda_{21}(s) & \Lambda_{22}(s) & \cdots & \Lambda_{2n}(s) \\
			\vdots & \vdots & \ddots & \vdots \\
			\Lambda_{n1}(s) & \Lambda_{n2}(s) & \cdots & \Lambda_{nn}(s)
		\end{bmatrix}
	\end{equation}
	Na osnovu izraza \eqref{eq:pojac_lambda}, možemo zaključiti da je :
	
	\begin{equation}
		RGA(G(s)) = G(s) \times G^{-1}(s)^T
	\end{equation}
	
	\noindent gde x označava element-po-element množenje matrica (Schurov proizvod). \\
	\clearpage
	
	Sam značaj RGA matrice i njena primena će se videti na projektovanju decentralizovanog upravljanja gde je osnovna ideja da se od MIMO sistema dobije n nezavisnih SISO kontura, odnosno gde će jedan upravljački signal regulisati samo jedna izlazni signal. Tada matrica RGA može biti od pomoći u izboru odgovarajućeg uparivanja upravljačkih i izlaznih signala. Odnosno ako upravljanje $u_j$ reguliše izlay $y_i$, tada u idealnom slučaju želimo da član $\lambda_{ij}(j\omega)$ bude jednak jedinici na svim učestanostima, čime se obezdeđuje da interakcija datog kanala sa preostalim kanalima bude minimalna, u idelanom slučaju da nepostoji. Samom izboru uparivanja signala upravljanja i izlaza ćemo se posvetiti više pri decentralizovanom upravljanju. \\
	
	Videli smo da se na osnovu RGA matrice može odrediti najbolji način uparivanja za decentralizovana upravljanje na željenoj poziciji. Uparivanje ulaza i izlaza možemo predstaviti preko matrica \textbf{C}, gde su na mestu odabranih parova $u_j \rightarrow y_i$ upisane 1, a na ostalim 0. Ukupan broj mogućih kombinacija matrice \textbf{C} je n!, što je u našem slučaju 2. 
	
	Dijagonalno uparivanje :
	
	\begin{equation}
		C = \begin{bmatrix}
			1 & 0 \\
			0 & 1 \\
		\end{bmatrix}
	\end{equation}
	
	Vandijagonalno uparivanje :
	
	\begin{equation}
		C = \begin{bmatrix}
			0 & 1 \\
			1 & 0 \\
		\end{bmatrix}
	\end{equation}
	
	Svakoj RGA matrici možemo pridružiti RGA broj koja opisuje odstupanje RGA matrice od idealne C matrice. 
	
	\begin{equation}
		RGA_{number}(\omega) = \left\| R_{GA}(G(j\omega)) - C \right\|_{sum}
	\end{equation}
	
	Kao "sto je prikazano na slici (Sl. \ref{fig:RGA}) vidimo da je za sve u"cestanosti dijagonalno uparivanje bolje, "stavi"se stepen dijagonalne dominantnosti je jasno izra"zen (razlika RGA brojeva iznosi vi"se od 4). Mo"zemo primetiti da bi nas analiza samo vrednosti RGA matrice dovela na pogre"san zaklju"cak, razlog za ovo je to "sto su vrednosti RGA matrice kompleksni brojevi, pa dejstvo faze ne vidimo kroz intezitet. Tako\dj{}e treba napomenuti da je relativno poja"canje na nultoj u"cestanosti negativno kod vandijagonalnog uparivanja, tako da to predstavlja jo"s jedan razlog za odbacivanje istog.
	
	\begin{figure}[!h]
		\centering
		\includegraphics[width=0.55\linewidth]{slike/zad1/RGA.eps}
		\caption{Prikaz vrednosti RGA matrice, kao i RGA broja}
		\label{fig:RGA}
	\end{figure}
	
	
	\newpage
	\subsection{Propusni opseg}
	
	Za multivarijabilni model objekta su snimljeni odzivi na nezavisne step pobude, rezultati se mogu videti na slici (Sl. \ref{fig:ol_step}).
	\vspace{1cm}
	
	\begin{figure}[!h]
		\centering
		\includegraphics[width=0.9\linewidth]{slike/zad1/Step_response_ol.eps}
		\caption{Prikaz odziva na nezavisne step pobude}
		\label{fig:ol_step}
	\end{figure}
	\vspace{1cm}
	
	Za svaku regulisanu promenljivu $y_i$ je odre\dj{}eno vreme uspona. Vrednosti su prikazane tabelarno (Tab. \ref{tab:uspon}).
	
	\begin{table}[h]
		\centering
		\begin{tabular}{|c|c|c|}
			\hline
			$T_r$ [s] & $y_1$  & $y_2$ \\ \hline
			$u_1$ & 3880   & 3750  \\ \hline
			$u_2$ & 3440   & 2560  \\ \hline
		\end{tabular}
		\caption{Procenjeno vreme uspona}
		\label{tab:uspon}
	\end{table}
	
	Imaju\'ci na umu (Tab. \ref{tab:uspon}) mo"zemo definisati najmanji dopustiv propusni opseg regulisanog sistema kao $w_{0\text{wc}} = 0.0016$ rad/s, a s obzirom na to da linearizovani model ne poseduje transmisione nule kao maksimalno ostvariv propusni opseg sistema mo"zemo usvojiti $w_0 = 0.0025$ rad/s, ovu vrednost usvajamo zato "sto odgovara najbr"zoj komponenti sistema, za o"cekivati je da je to mogu\'ce posti\'ci bez ugro"zavanja stabilnosti ceokupnog sistema. 
	
	
	
	
	\clearpage
	\subsection{Direkcionalnost upravljanja}
	
	Nad matricom funckije G(j$\omega$) moguće je izvršiti singularnu dekompoziciju, odnosno :
	\\
	
	\begin{equation}
		\mathbf{G = U\Sigma V^T}
	\end{equation}
	
	\noindent Gde U predstavlja vektor direkcionalnosti ulaza , V vektor direkcionalnosti izlaza  i $\Sigma$ matricu pojačanja sistema, odnosno dijagonalnu matricu singularnih vrednosti u nerastućem poretku. \\
	\begin{equation}
		\mathbf{\Sigma} = \begin{bmatrix}
			\sigma_1 = \bar{\sigma} & 0 & \cdots & 0 \\
			0 & \sigma_2 & \cdots & 0 \\
			\vdots & \vdots & \ddots & \vdots \\
			0 & 0 & \cdots & \sigma_r = \underline{\sigma} \\
		\end{bmatrix}
	\end{equation}
	
	\noindent Singularnu dekompoziciju ćemo izvršiti nad G(j$\omega$) na nultoj učestanosti, na učestanosti propusnog opsega i na visokim učestanostima (van propusnog opsega). \\\\
	
	Na slikama (Sl. \ref{fig:direkc_1}), (Sl. \ref{fig:direkc_2}) i (Sl. \ref{fig:direkc_3}) mo"zemo videti  pravce 
	vektora ulaza i izlaza pridru"zene najve\'cem i najmanjem pojačanju za navedene učestanosti. Mo"zemo primetiti da je multivarijabilnost objekta upravljanja jako izra"zena, razlike u poja"canjima iznose i do 100 dB.\\
	
	\begin{figure}[!ht]
		\centering
		\includegraphics[width=0.7\linewidth]{slike/zad1/RGA_1.eps}
		\caption{Pravci vektora ulaza i izlaza}
		\label{fig:direkc_1}
	\end{figure}
	
	
	
	
	\begin{figure}[!ht]
		\centering
		\includegraphics[width=0.7\linewidth]{slike/zad1/RGA_2.eps}
		\caption{Pravci vektora ulaza i izlaza}
		\label{fig:direkc_2}
	\end{figure}
	
	
	\begin{figure}[!ht]
		\centering
		\includegraphics[width=0.7\linewidth]{slike/zad1/RGA_3.eps}
		\caption{Pravci vektora ulaza i izlaza}
		\label{fig:direkc_3}
	\end{figure}
	
	
	\clearpage
	
	\section{Projektovanje sistema upravljanja}
	
	Radi uniformnosti komparativne analize te"zimo slede\'cim projektnim zadacima: 
	\begin{adjustwidth}{2em}{2em}
		\begin{itemize}
			\item $\omega_0 = 0.002 \frac{\text{rad}}{\text{s}}$ 
			\item $\Phi_{pf} = 84.3^{\circ}$
			\item $M_s \approx 1 $
		\end{itemize}
	\end{adjustwidth}
	\noindent Razlog za ovaj izbor parametara je slede\'ci: Kako jedan on kontrolera na bazi poptpune inverzije dinamike i na"s objekat upravljanja ne poseduje neminimalne nule onda su veli"cine $\omega_0$ i $\Phi_{pf}$ (odnosno  $M_s$) direktno povezane, tj. imamo samo 1 stepen slobode za izbor parametara. Kao $\omega_0$ usvajamo $0.002~ \frac{\text{rad}}{\text{s}}$ zato "sto ta u"cestanost predstavlja kompromis izme\dj{}u brzine i robustnosti sistema, efektivno ubrzavamo najsporiji proces i malo usporavamo br"zi proces.\\
	
	
	Takođe, kao što smo objasnili potrebu za uniformošću, izvršićemo skaliranje matrice funkcije prenosa objekta, tako da postupkom projektovanja dobijamo normalizovani kontroler, koji je na kraju potrebno deskalirati. Odnosno vr"simo slede\'ci postupak:
	\begin{align}
		G(s) &= D^{-1}_y\hat{G}(s)D_u \\
		\hat{K}(s) &= D_uK(s)D^{-1}_y
	\end{align}
	\noindent Nakon "sto isprojektujemo normalizovani kontroler $K(s)$ za dati normalizovani objekat upravljanja $G(s)$ vr"simo evaluaciju performansi nad normalizovanom matricom funkcija osetljivosti $S(s)$, a ne $\hat{S}(s)$, razlog za ovaj izbor je to "sto korišćenje matrice funkcija prenosa osetljivosti $\hat{S}(s)$
	nema puno smisla u analizi kvaliteta praćenja reference kontrolera $\hat{K}(s)$, nad neskaliranim 
	objektom upravljanja $\hat{G}(s)$, pošto pojedinačni signali vektora reference i 
	vektora greške praćenja ne moraju uopšte biti uporedivih amplituda.
	\begin{figure}[!h]
		\centering
		\includegraphics[width = 0.8\linewidth]{slike/zad2/model.pdf}
		\caption{Simulink model za komparativnu analizu }
		\label{fig:model_sim}
	\end{figure}
	\clearpage
	\subsection{Decentralizovani kontroler}
	Ideja decentralizovanog upravljanje jeste projektovanje kontrolera za svaki kanal regulacije ponaosob, odnosno projektujemo kontrolere za prenos $r_j \rightarrow y_i$ pretpostavljajući da \'ce neželjena spregnutost me\dj{}u kanalima biti zanemarljiva. Zapravo cilj je da se od MIMO sistema dobije N nezavisnih SISO kontura, gde će jedan upravljački signal regulisati samo jedan izlazni signal i za svaku konturu isprojektovati SISO kontroler. Već smo spomenuli da RGA matrica ili RGA broj igra značajnu ulogu pri projektovanju decentralizovanog kontrolera. Op"ste pravilo pri izboru kanala regulacije ka"ze da obacujemo ona oparivanja $u_j \rightarrow y_i$ za koje je $\lambda_{ij}(j0) < 0$ i $\lambda_{ij}(j\omega_0) < 0$.\\
	
	Nakon izabranog uparivanja, formira se matrica uparivanja \textbf{C} koja sadrži 1 na mestu odabranih parova za uparivanje, a na ostalim 0. Matrica uparivanja \textbf{C} se može upotrebiti za dijagonalizaciju objekta upravljanja \textbf{G}(s) primenom slede\'ce transformacije:
	\begin{equation}
		\mathbf{G}^*(s) = \mathbf{G}(s) \mathbf{C}^{-1}
	\end{equation}
	gde je $\mathbf{G}^*(s)$ forme pogodne za decentralizovano upravljanje, odnosno na svojoj glavnoj dijagonali sadrži funkcije prenosa koje su izabrane kao kanali upravljanja.\\
	
	Kontroler koji se projektuje za ovakav objekta upravljanja je dijagonalan i na dijagonali se nalaze pojedinačni SISO kontroleri:
	
	\begin{equation}
		\mathbf{K}^*(s) = \text{diag}(\mathbf{K}_1^*(s), \mathbf{K}_2^*(s), \dots, \mathbf{K}_n^*(s)) =
		\begin{bmatrix}
			\mathbf{K}_1^*(s) & 0 & \cdots & 0 \\
			0 & \mathbf{K}_2^*(s) & \ddots & \vdots \\
			\vdots & \ddots & \ddots & 0 \\
			0 & \cdots & 0 & \mathbf{K}_n^*(s) \\
		\end{bmatrix}
	\end{equation}
	
	$\mathbf{K}^*(s)$ reguliše $\mathbf{G}^*(s)$, a finalni kontroler koji regulise G(s) je K(s):
	
	\begin{equation}
		\mathbf{K}(s) = \mathbf{C}^{-1} \mathbf{K}^*(s)
	\end{equation}
	
	\begin{figure}[!h]
		\centering
		\begin{subfigure}[b]{0.45\textwidth}
			\includegraphics[width=\linewidth]{slike/zad2/dec1.png}
			\caption{  }
			\label{fig:dec1}
		\end{subfigure}
		\hfill
		\begin{subfigure}[b]{0.45\textwidth}
			\includegraphics[width=\linewidth]{slike/zad2/dec2.png}
			\caption{}
			\label{fig:dec2}
		\end{subfigure}
		\hfill
		\caption{ Dijagonalizovan objekat upravljanja 
			a) dijagonalizacija originalnog objekta, b) finalni kontroler}
		
	\end{figure}
	
	Neka je $\hat{\textbf{G}}(s)$ idealno dijagonalizovan objekat upravljanja, odstupanje od stvarnog modela se mo"ze prikazati kao paralelna $\hat{\textbf{G}}(s)$ i $\mathbf{\Delta}(s)$. Iz jednakosti: $\textbf{G}(s) = \hat{\textbf{G}}(s) + \hat{\textbf{G}}(s)\mathbf{\Delta}(s)$ se mo"ze odrediti izraz za $\mathbf{\Delta}(s)$. Imaju\'ci ovo na umu, mo"zemo odrediti uticaj odstupanja od savr"sene dijagonalizacije na degradaciju ostvarenih performansi kao: 
	\begin{equation}\label{eq:degragacija_perf}
		\textbf{S}(j\omega) = \hat{ \textbf{S}}(j\omega)(\textbf{I}+\hat{\textbf{T}}(j\omega)\mathbf{\Delta}(j\omega))^{-1}
	\end{equation}
	Gde su $\hat{ \textbf{S}}(j\omega)$ i $\hat{\textbf{T}}(j\omega)$ performanse projektovanog kontrolera nad savr"seno dijagonalizovanom objektu upravljanja.
	
	
	\newpage
	\subsubsection{Projektovanje kontrolera}
	
	Pored metode nezavisnog kanalnog projektovanja, za projektovanje decentralizovanog upravljanje moguće je sprovesti i metodu sekvencijalnog projektovanja, gde projektovanje kreće od najnezavisnije konture. Što se tiče sistema 2x2 nema razlike. Takođe, za sistema 2x2, moguća su samo dva tipa uparivanja, dijagonalno i vandijagonalno, odnosno, matrica C u našem slučaju može imati 2 oblika. RGA analiza nam služi kako bismo odlučili šta je bolje. U opštem slučaju, izbor uparivanje zavisi od učestanosti propusnog opsega sistema za koji projektujemo kontroler, odnosno da u nekom opsegu učestanosti vandijagonalno uparivanje daje bolje rezultate, a u drugom dijagonalno. Na osnovu RGA broja (Sl. \ref{fig:RGA}) možemo zaključiti da bolje rezultate dobijamo sa dijagonalnim uparivanjem na celom skupu učestanosti. Dodatno, računanjem RGA vrednosti na nultoj učestanosti dobijamo da su vandijagonalni elementi negativni, te se odlučujemo za dijagonalno uparivanje, odnosno, matrica C glasi:
	
	\begin{equation}
		C = \begin{bmatrix}
			1 & 0 \\
			0 & 1 \\
		\end{bmatrix}
	\end{equation}
	
	Nakon skaliranja matrice funckije prenosa koristeći izraz \eqref{eq:skaliranje}, dobijamo dva normalizovana kanala regulacije predstavljena funkcijama prenosa:
	
	\begin{equation}
		G^*_{11}(s) = G_{11}(s) = \frac{0.001388}{{ s + 0.0002778}}
	\end{equation}
	
	\begin{equation}
		G^*_{22}(s) = G_{22}(s) = \frac{0.002183 s + 8.336*10^{-7}}{{  s^2 + 0.0008889 s + 1.698*10^{-7}}}
	\end{equation}
	\\\\
	Za oba kanala projektujemo PI kontrolere za zahtevanim propusnim opsegom od 0.02 rad/s i pretekom faze od 83,7 $^{\circ}$. Na osnovu traženih zahteva, najpre dobijamo dva SISO kontrolera za normalizovani objekat upravljanja. Nakon toga, izvršeno je deskaliranje kontrolera, i dobijamo: 
	
	\begin{equation}
		\hat{K}^*_{11}(s) = \frac{8.564*10^4 s + 41.46}{{ s}}
	\end{equation}
	
	\begin{equation}
		\hat{K}^*_{22}(s) = \frac{0.02696 s + 1.94*10^{-5}}{{ s}}
	\end{equation}
	\\\\
	Kako smo izabrali dijagonalno uparivanje, dobijamo da je finalni kontroler : 
	\\
	
	\begin{equation}
		\hat{\textbf{K}}(s) = \begin{bmatrix} 
			\dfrac{8.564 \times 10^4 s + 41.46}{s} & 0 \\
			0 & \dfrac{0.02696 s + 1.94 \times 10^{-5}}{s}
		\end{bmatrix}
	\end{equation}
	
	\newpage
	
	Na slici (Sl. \ref{fig:sigmaDEC}) su prikazane ostvarene singularne karakteristike funkcije osetljivosti i komlementarne osetljivosti regulisanog sistema, dok su na slici (Sl. \ref{fig:sigmaG_W_dec}) prikazane singularne karakteristike originalnog i regulisanog sistema. Za prikaz karakteristika su korišćene normalizovane matrice.
	
	\hspace{2cm}
	
	\begin{figure}[h]
		\centering
		\begin{subfigure}{0.6\linewidth}
			\centering
			\includegraphics[width=\linewidth]{slike/zad2/sigma_K_decent_G_tf_norm.eps}
			\caption{Karakteristike osetljivosti i komp. osetljivosti}
			\label{fig:sigmaDEC}
		\end{subfigure}
		\hfill
		\begin{subfigure}{0.6\linewidth}
			\centering
			\includegraphics[width=\linewidth]{slike/zad2/Sigma_G_W_decenent.eps}
			\caption{Originalni sistem i funkcija povratnog prenosa}
			\label{fig:sigmaG_W_dec}
		\end{subfigure}
		\caption{Singularne karakteristike, decentalizovano upravljanje}
	\end{figure}
	
	\hspace{2cm}
	
	Sa slike (Sl. \ref{fig:sigmaDEC}) vidimo da se gornja i donja singularna karakteristika osetljivosti spajaju negde oko $10^{-2}$ rad/s, u blizini željenog propusnog opsega. Međutim, pre toga, primetno je ogromno razilaženje gornje i donje singularne karakteristike. Na osnovu funkcije osetljivosti možemo odrediti ostvareni propusni opseg učestanosti i on iznosi $\omega_0$ = 6.3148*$10^{-4}$, što je značajno manje od željenog. Međutim, za vršne vrednosti osetljivosti i komplementarne osetljivosti možemo reći da su oko preporučenih vrednosti, $M_S$ = 1.0010, $M_T$ = 1.0313. Kao "sto vidimo na (Sl. \ref{fig:sigmaG_W_dec}) gornja i donja singularna karakteristika funkcije spregnutog prenosa su razdvojene na celom opsegu učestanosti.
	
	
	\clearpage
	\begin{figure}[!ht]
		\centering
		\includegraphics[width=0.7\linewidth]{slike/zad2/step_T_decent.eps}
		\caption{Zavisnost izlaza od vremena na jediničnu step pobudu}
		\label{fig:stepT_dec}
	\end{figure}
	
	\begin{figure}[!ht]
		\centering
		\includegraphics[width=0.7\linewidth]{slike/zad2/step_KS_decent.eps}
		\caption{Zavisnost upravljanja od vremena na jediničnu step pobudu}
		\label{fig:stepKS_dec}
	\end{figure}
	
	
	
	Na slici (Sl. \ref{fig:stepT_dec}) prikazani su odzivi na step pobudu. Dijagonalni grafici su zadovoljavajući, međutim vandijagonalni grafici govore da ipak postoji izra"zeno sprezanje medju kanalima. Odnosno, sa ovakivim pristupom nismo uspeli da suzbijemo interakciju među kanalima. Zbog toga decentralizovano upravljanje u velikom broju slučajeva ne daje dobre rezultate. Na lici (Sl. \ref{fig:stepKS_dec}) mo"zemo videti step odziv upravljanja, u pitanju su upravljanja prihvatljivih amplituda.
	
	\newpage
	\subsection{Dekuplujuće upravljanje sa statičkim dekulperom}
	
	Ideja pri projektovanju decentralizovanog upravljanja bila je da izborom matrice \textbf{C} obezbedi dijagonalizovani objekat upravljanja koji će biti pogodan za projektovanje nezavisnih SISO kontrolera. Međutim, ovakvim pristupom ne poku"savamo da akvtivnim dejstvom suzbijemo iznetrakcije, ve\'c se pouzdamo u mali stepen interakcije me\dj{}u kanalima. \\
	
	U cilju postizanja boljih performansi i smanjenju neželjenih interakcija spregnutih kanala regulacije, uvode se pretkompezator $W_1$ i postkompenzator $W_2$ (Sl. \ref{fig:dek}) "ciji je zadatak da aktivnim dejtvom vr"se potiskivanje interakcije me\dj{}u kanalima.
	
	
	\begin{figure}[!h]
		\centering
		\includegraphics[width=0.6\linewidth]{slike/zad2/dekup.jpg}
		\caption{Dekuplovanje objekta upravljanja}
		\label{fig:dek}
	\end{figure}
	
	
	Cilj je ostvariti objekat u dijagonalnoj formi : 
	
	\begin{equation}
		\begin{aligned}
			G^*(s) &= W_2(s) G(s) W_1(s) 
			&\approx
			\begin{bmatrix}
				G_{11}^*(s) & 0 & \dots & 0 \\
				0 & G_{22}^*(s) & \ddots & \vdots \\
				\vdots & \ddots & \ddots & 0 \\
				0 & \dots & 0 & G_{nn}^*(s)
			\end{bmatrix} ,
		\end{aligned}
		\label{eq:dijag}
	\end{equation}
	
	za koji se može projektovati dijagonalni kontroler $K^*$(s) koji se sastoji od N SISO kontrolera. \\
	
	Kako se originalni izlazni vektor $y$ ne poklapa sa redefinisanim $y^*$, pri dekuplovanju se ne koristi postkomenzator $W_2$, čime dobijamo da je 
	
	\begin{equation}
		\mathbf{G}^*(s) = \mathbf{G}(s) \mathbf{W_1},
	\end{equation}
	
	\noindent dok konačni kontroler glasi :
	
	\begin{equation}
		\mathbf{K}(s) = \mathbf{W_1}(s) \mathbf{K}^{*}
	\end{equation}
	
	\noindent Mi ćemo razmotriti dve vrste statičkih dekuplera : dekupler na nultoj učestanosti(dekupler ustaljenog stanja) i dekupler na učestanosti propusnog opsega. Samim tim, ovakvim dekuplerima se može ostvariti dekuplovanje samo na nekom opsegu učestanosti. 
	
	
	\newpage
	\subsubsection{Dekupler ustaljenog stanja}
	
	U slučaju da sistem ne poseduje astatizam moguće je izvršiti dekuplovanje na učestanosti $\omega$ = 0, koristeći pretkompenzator $\textbf{W}_1$:
	\begin{equation}
		\textbf{W}_1(s) = \textbf{W}_1 = \textbf{G}^{-1}(0)
	\end{equation}
	
	\vspace{1cm}
	Konkretno, u našem slučaju, dobija se da je $W_1$ : 
	
	\begin{equation}
		\textbf{W}_1 = \begin{bmatrix} 0.5045  &  0.5133 \\  0.3044  &  0.5133 \end{bmatrix}
	\end{equation}
	
	\vspace{1cm}
	Sada, dekuplovani objekat upravljanja izgleda: 
	
	\begin{equation}
		\textbf{G}^*(s) = \begin{bmatrix} 
			\dfrac{0.0002778s + 7.716 \times 10^{-8}}{s^2 + 0.0005556s + 7.716 \times 10^{-8}} & 0 \\
			0 & \dfrac{0.0006111s^3 + 7.13 \times 10^{-7}s^2 + 2.546 \times 10^{-10}s + 2.882 \times 10^{-14}}{s + 0.005147}
		\end{bmatrix}
	\end{equation}
	
	
	U slučaju ustaljenog stanja (s = 0), vidimo da matica $\textbf{G}^*$(s) postaje jedinična matrica, na šta smo i ciljali. \\
	
	\vspace{1cm}
	Zahtevi za projektovanje kontrolera ostaju isti kao u slučaju decentralizovanog upravljanja, čime dobijamo :
	
	\begin{equation}
		\textbf{K}^*(s) = \begin{bmatrix} 
			\dfrac{7.065s + 0.00342}{s} & 0 \\
			0 & \dfrac{3.157s + 0.00264}{s}
		\end{bmatrix}
	\end{equation}
	
	\vspace{1cm}
	\noindent Finalni kontroler glasi: 
	
	\begin{equation}
		\hat{\textbf{K}}(s) = \begin{bmatrix}  
			\dfrac{1.581 \times 10^5s + 76.55}{s} & \dfrac{1.595 \times 10^5s + 77.19}{s} \\
			\dfrac{0.02221s + 1.858 \times 10^{-5}}{s} & \dfrac{0.04887s + 4.087 \times 10^{-5}}{s}
		\end{bmatrix}
	\end{equation}
	
	
	Možemo primetiti da finalni kontroler nije dijagonalan, već je multivarijabilan. 
	
	\newpage
	
	Na slici (Sl. \ref{fig:sigmaDEK0}) su prikazane ostvarene singularne karakteristike funkcije osetljivosti i komlementarne osetljivosti regulisanog sistema, dok su na slici (Sl. \ref{fig:sigmaG_W_dek0}) prikazane singularne karakteristike originalnog i regulisanog sistema. Za prikaz karakteristika su korišćene normalizovane matrice.
	
	\hspace{2cm}
	
	\begin{figure}[h]
		\centering
		\begin{subfigure}{0.6\linewidth}
			\centering
			\includegraphics[width=\linewidth]{slike/zad2/sigma_KG_tf_norm_decoupling_dc.eps}
			\caption{Karakteristike osetljivosti i komp. osetljivosti}
			\label{fig:sigmaDEK0}
		\end{subfigure}
		\hfill
		\begin{subfigure}{0.6\linewidth}
			\centering
			\includegraphics[width=\linewidth]{slike/zad2/Sigma_G_W_decopling_dc.eps}
			\caption{Originalni sistem i funkcija povratnog prenosa}
			\label{fig:sigmaG_W_dek0}
		\end{subfigure}
		\caption{Singularne karakteristike, dekupluju\'ce upravljanje na nultoj u"cestanosti}
	\end{figure}
	
	\hspace{2cm}
	
	
	
	Na osnovu slika (Sl. \ref{fig:sigmaDEK0}, Sl. \ref{fig:sigmaG_W_dek0} ) možemo zaključiti da su gornje i donje singularne karakteristike bliže jedna drugoj, i da time dobijamo bolje rezultate u odnosu na decentralizovano upravljanje. Ostareni propusni opseg je sada $\omega_0$ = 0.0013, što je i dalje od željenog.
	
	\newpage
	
	\begin{figure}[!ht]
		\centering
		\includegraphics[width=0.7\linewidth]{slike/zad2/step_T_decoupling_dc.eps}
		\caption{Zavisnost izlaza od vremena na jediničnu step pobudu}
		\label{fig:stepT_dek0}
	\end{figure}
	
	\begin{figure}[!ht]
		\centering
		\includegraphics[width=0.7\linewidth]{slike/zad2/step_KS_decoupling_dc.eps}
		\caption{Zavisnost upravljanja od vremena na jediničnu step pobudu}
		\label{fig:stepKS_dek0}
	\end{figure}
	
	
	Na slici (Sl. \ref{fig:stepT_dek0}) prikazani su odzivi na step pobudu. Dijagonalni grafici su zadovoljavajući, na vandijagonalnim mo"zemo videti da i dalje postoji sprega, ali da se kontroler uspe"sno bori protiv iste. Odnosno, sa ovakivim pristupom smo uspeli da suzbijemo interakciju među kanalima. Na lici (Sl. \ref{fig:stepKS_dek0}) mo"zemo videti step odziv upravljanja, u pitanju su upravljanja prihvatljivih amplituda.
	
	
	\newpage
	
	
	\subsubsection{Dekupler na učestanosti propusnog opsega}
	
	Kao što smo već rekli, efekat dekuplovanja je vidljiv samo na nekom opsegu učestanosti, pa je ideja izvršiti dekuplovanje na nekoj u učestanosti oko propusnog opsega. 
	Kako je \textbf{G}(j$\omega_0$) matrica kompleksnih brojeva, nije moguće napraviti njenu potpunu inverziju, već se koristi realna aproksimacija inverzije matrice. Odnosno, statički dekupler glasi: 
	
	\begin{equation}
		\textbf{W}_1(s) = \textbf{W}_1 = c \cdot \textbf{G}^{-1}(j\omega_0), 
	\end{equation}
	
	\noindent gde je \textit{c} neki kompleksni broj jediničnog modula koji obezbeđuje da matrica $\textbf{W}_1$ ima sve realne koeficijente. \\
	
	\vspace{1cm}
	Realnom inverzijom na učestanosti $\omega_0$, dobija se dekupler:
	
	\begin{equation}
		\textbf{W}_1 = \begin{bmatrix} 2.6701  &  1.7566 \\  1.2240  &  1.7566 \end{bmatrix},
	\end{equation}
	
	\vspace{1cm}
	\noindent kojim se ostvaruje dekuplovani objekat upravljanja: \\
	
	\begin{equation}
		\textbf{G}^*(s) = \begin{bmatrix} 
			\dfrac{0.002007s + 5.575 \times 10^{-7}}{s^2 + 0.0005556s + 7.716 \times 10^{-8}} & 0 \\
			0 & \dfrac{0.002091s^3 + 2.44 \times 10^{-6}s^2 + 8.714 \times 10^{-10}s + 9.861 \times 10^{-14}}{s^4 + 0.001778s^3 + 1.13 \times 10^{-6}s^2 + 3.018 \times 10^{-10}s + 2.882 \times 10^{-14}}
		\end{bmatrix}
	\end{equation}
	
	\vspace{2cm}
	Kao i u prethodna dva slučaju, za dijagonalizovani objekat $G^*$(s) projektuje se decentralizovani kontoler sa istim zahtevima za propusni opseg i pretek faze. Dobija se dijagonalni kontroler :
	
	\begin{equation}
		\textbf{K}^*(s) = \begin{bmatrix} 
			\dfrac{0.9778s + 0.0004734}{s} & 0 \\
			0 & \dfrac{0.9226s + 0.0007715}{s}
		\end{bmatrix}
	\end{equation}
	
	\vspace{1cm}
	\noindent Finalni kontroler glasi: 
	
	\begin{equation}
		\hat{\textbf{K}}(s) = \begin{bmatrix}  
			\dfrac{2.611s + 0.001264}{s} & \dfrac{1.197s + 0.0005794}{s} \\  
			\dfrac{1.621s + 0.001355}{s} & \dfrac{1.621s + 0.001355}{s} 
		\end{bmatrix}
	\end{equation}
	
	\vspace{1cm}
	\noindent Kao i u prethodnom slučaju, finalni kontroler je multivarijabilan. 
	
	\newpage
	
	Na slici (Sl. \ref{fig:sigmaDEKw0}) su prikazane ostvarene singularne karakteristike funkcije osetljivosti i komlementarne osetljivosti regulisanog sistema, dok su na slici (Sl. \ref{fig:sigmaG_W_dekw0}) prikazane singularne karakteristike originalnog i regulisanog sistema. Za prikaz karakteristika su kor"s\'cene normalizovane matrice.
	
	\hspace{2cm}
	
	\begin{figure}[h]
		\centering
		\begin{subfigure}{0.6\linewidth}
			\centering
			\includegraphics[width=\linewidth]{slike/zad2/sigma_KG_tf_norm_decoupling_w0.eps}
			\caption{Karakteristike osetljivosti i komp. osetljivosti}
			\label{fig:sigmaDEKw0}
		\end{subfigure}
		\hfill
		\begin{subfigure}{0.6\linewidth}
			\centering
			\includegraphics[width=\linewidth]{slike/zad2/Sigma_G_W_decoupling_w0.eps}
			\caption{Originalni sistem i funkcija povratnog prenosa}
			\label{fig:sigmaG_W_dekw0}
		\end{subfigure}
		\caption{Singularne karakteristike, dekupluju\'ce upravljanje na u"cestanosti propusnog opsega}
	\end{figure}
	
	\hspace{2cm}
	
	Sa slike (Sl. \ref{fig:sigmaDEKw0}) vidimo da se gornja i donja singularna karakteristika osetljivosti spajaju u okolini blizini željenog propusnog opsega. Tako\dj{}e, pre toga, stepen razilaženje gornje i donje singularne karakteristike je mali. Na osnovu funkcije osetljivosti možemo odrediti ostvareni propusni opseg učestanosti i on iznosi $\omega_0 = 0.0017 \frac{\text{rad}}{\text{s}}$. Međutim, za vršne vrednosti osetljivosti i komplementarne osetljivosti možemo reći da su oko preporučenih vrednosti, $M_S$ = 1.0000, $M_T$ = 1.0477. Kao "sto vidimo na (Sl. \ref{fig:sigmaG_W_dekw0}) gornja i donja singularna karakteristika funkcije spregnutog prenosa su razmaknute na u"cestanostima manjim od propusnog opsega, na vi"sim u"cestanostima su spojene.
	
	
	
	
	\newpage
	
	\begin{figure}[!ht]
		\centering
		\includegraphics[width=0.7\linewidth]{slike/zad2/step_T_decoupling_w0.eps}
		\caption{Zavisnost izlaza od vremena na jediničnu step pobudu}
		\label{fig:stepT_dekW0}
	\end{figure}
	
	\begin{figure}[!ht]
		\centering
		\includegraphics[width=0.7\linewidth]{slike/zad2/step_KS_decoupling_w0.eps}
		\caption{Zavisnost upravljanja od vremena na jediničnu step pobudu}
		\label{fig:stepKS_dekW0}
	\end{figure}
	
	Na slici (Sl. \ref{fig:stepT_dekW0}) prikazani su odzivi na step pobudu. Dijagonalni grafici su zadovoljavajući, na vandijagonalnim mo"zemo videti da je sprega me\dj{}u kanalima skoro u potpunosti potisnuta. Uspe"sno smo suzbili interakciju među kanalima. Na lici (Sl. \ref{fig:stepKS_dekW0}) mo"zemo videti step odziv upravljanja, u pitanju su upravljanja prihvatljivih amplituda.
	
	
	\newpage
	
	
	\newpage
	\subsection{Dekuplujuće upravljanje sa dinamičkim dekuplerom}
	
	Videli smo da se projektovanje statičkim dekuplerima ostavaruje dekuplovanje samo na nekom opsegu učestanosti. Radi postizanja efekta dekuplovanja na što širem opsegu učestanosti koristi se dinamičko dekuplovanje. \\
	
	Za razliku od statičkih dekuplera, gde je pretkompenzator statička matrica, kod dinamičkih dekuplera $\textbf{W}_1(j\omega)$ je dinamička matrica. Postoje dve vrste dinamičkog dekuplovanja, a to su dinamičko dekuplovanje na bazi inverzije dinamike i dinamičko dekuplovanje bez potpune inverzije dinamike. Drugi pristup automatizuje postupak uspostavljanja kauzalnosti kontrolera, pa je to i njegova prednost. Mi smo se odlučili za projektovanje kontrolera sa potpunom inverzijom dinamike. Jedna od mana ovakvog tipa projektovanja jeste što je red kontrolera je jednak redu samog objekta upravljanja. Tako\dj{}e ovaj kontroler vrlo "cesto rezultuje neprihvatljivim intezitetima upravljanja. \\
	
	
	
	\subsubsection{Kontroler na bazi inverzije dinamike}
	
	Posmatrajući izraz \eqref{eq:dijag}, najpogodnija situacija je u slučaju kada je matrica $\textbf{G}^*$(s) jednaka jediničnoj matrici, čime se obezbeđuje da interakcije jednog kanala sa drugima ne postoje. Na osnovu toga, dobija se da je  :
	
	\begin{equation}
		\textbf{W}_1(s) = \textbf{G}^{-1}(s)
	\end{equation}
	
	Za decentralizovani jedinični objekat, projektuje se decentralizovani kontroler za dekuplujuće kanale: 
	
	\begin{equation}
		\mathbf{K}^*(s) = \text{diag}({K}_1^*(s), {K}_2^*(s), \dots, {K}_n^*(s)) , 
	\end{equation}
	
	\noindent na osnovu koga se dobija multivarijabilni (idealni) invertor dinamike: 
	
	\begin{equation}
		\textbf{K}(s) = \textbf{W}_1(s) \textbf{K}^*(s) = \textbf{G}^{-1}(s) \cdot \mathrm{diag}(\frac{\omega_{01}}{s}, \frac{\omega_{02}}{s}, \dots, \frac{\omega_{0n}}{s}), 
	\end{equation}
	
	\noindent gde $\omega_{0i}$ predstavlja propusni opseg i-tog kanala regulacije.\\
	
	Veoma često je potrebno uspostaviti kauzalnost kontrolera dodavanjem fitra reference dovoljno velikog stepena $n_p$:
	
	\begin{equation}
		\textbf{K}(s) = \textbf{W}_1(s) \textbf{K}^*(s) \textbf{F}(s), \quad \textbf{F}(s) = \frac{1}{{(\frac{s}{\omega_{pk}} + 1)}^{n_p}}
	\end{equation}
	
	Pored uspostavljanja kauzalnosti, isto kao u slučaju SISO sistema, ograničenja za realizaciju kontrolera na bazi inverzije dinamike su da sistem bude stabilan i da nema neminimalno-fazne nule. \\
	\clearpage
	
	Naš normalnizovani objekat upravljanja u okolini nominalnog režima glasi: 
	
	%\begin{equation}
	%G^*(s) = \begin{bmatrix} \frac{{ 0.001388}}{{s + 0.0002778}} %&  \frac{{-0.0009921 s - 5.029*10^{-7}}}{s^2 + 0.0008889 s + %1.698*10^{-7}}   \\  -\frac{{ 0.001388}}{{s + 0.0002778}} &  %\frac{{0.002183 s + 8.336*10^{-7}}}{s^2 + 0.0008889 s + %1.698*10^{-7}} \end{bmatrix}
		%\end{equation}
		
		\begin{equation}
			\textbf{G}^*(s) = \begin{bmatrix} 
				\dfrac{0.001388}{s + 0.0002778} & \dfrac{-0.0009921s - 5.029 \times 10^{-7}}{s^2 + 0.0008889s + 1.698 \times 10^{-7}} \\
				-\dfrac{0.001388}{s + 0.0002778} & \dfrac{0.002183s + 8.336 \times 10^{-7}}{s^2 + 0.0008889s + 1.698 \times 10^{-7}}
			\end{bmatrix}
		\end{equation}
		\vspace{1cm}
		
		Za ovakav objekat upravljanja, dekupler na bazi inverzije dinamike je:
		
		%\begin{equation}
		%W_1 = \begin{bmatrix} \frac{{ 26.42 s^2 + 0.01743 s + %2.803*10^{-6}}}{{s^2 + 0.02028 s + 5.556*10^{-6}}}   &   %\frac{{ 12.01 s^2 + 0.009424 s + 1.691*10^{-6}}}{{s^2 + %0.02028 s + 5.556*10^{-6}}} \\  \frac{{ 16.8 s^2 + 0.01493 s %+ 2.852*10^{-6}}}{{s^2 + 0.02028 s + 5.556*10^{-6}}}  &  %\frac{{ 16.8 s^2 + 0.01493 s + 2.852*10^{-6}}}{{s^2 + %0.02028 s + 5.556*10^{-6}}} \end{bmatrix},
		%\end{equation}
		\vspace{1cm}
		\begin{equation}
			\textbf{W}_1 = \begin{bmatrix} 
				\dfrac{{26.42s^2 + 0.01743s + 2.803 \times 10^{-6}}}{{s^2 + 0.02028s + 5.556 \times 10^{-6}}} & \dfrac{{12.01s^2 + 0.009424s + 1.691 \times 10^{-6}}}{{s^2 + 0.02028s + 5.556 \times 10^{-6}}} \\
				\dfrac{{16.8s^2 + 0.01493s + 2.852 \times 10^{-6}}}{{s^2 + 0.02028s + 5.556 \times 10^{-6}}} & \dfrac{{16.8s^2 + 0.01493s + 2.852 \times 10^{-6}}}{{s^2 + 0.02028s + 5.556 \times 10^{-6}}}
			\end{bmatrix}
		\end{equation}
		
		\vspace{1cm}
		Vidmo da je dekupler granično kauzalan. Kao i do sada, najpre projektujemo dva SISO PI kontrolera sa zathevanim propusnim opsegom od $\omega_0 = 0.002$  $\frac{\text{rad}}{\text{s}}$ i pretekom faze  $\Phi_{pf} = 84.3 ^{\circ}$. \\
		
		Nakon projektovanja, dobijamo da je dijagonalni kontroler :
		
		\begin{equation}
			\textbf{K}^*(s) = \begin{bmatrix} \frac{{ 0.002}}{s} & 0 \\ 0 & \frac{0.002}{s} \end{bmatrix}
		\end{equation}
		
		\vspace{1cm}
		\noindent Konačno, finalni kontroler glasi:
		
		
		%\begin{equation}
		%K(s) = \begin{bmatrix} \frac{{ 0.05284 s^2 + 3.486*10^{-5} s + 5.606*10^{-9}}}{{s^3 + %0.02028 s^2 + 5.556*10^{-6} s}}   &   \frac{{ 0.02402 s^2 + 1.885*10^{-5} s + %3.382*10^{-9}}}{{s^3 + 0.02028 s^2 + 5.556*10^{-6} s}} \\  \frac{{ 0.0336 s^2 + %2.987*10^{-5} s + 5.704*10^{-9}}} {{s^3 + 0.02028 s^2 + 5.556*10^{-6} s}}  &  \frac{{ %0.0336 s^2 + 2.987e-05 s + 5.704e-09}}{{{{s^3 + 0.02028 s^2 + 5.556*10^{-6} s}}}} %\end{bmatrix},
			%\end{equation}
			
			
			\begin{equation}
				\hat{\textbf{K}}(s) = \begin{bmatrix} \dfrac{{0.05284s^2 + 3.486 \times 10^{-5}s + 5.606 \times 10^{-9}}}{{s^3 + 0.02028s^2 + 5.556 \times 10^{-6}s}} & \dfrac{{0.02402s^2 + 1.885 \times 10^{-5}s + 3.382 \times 10^{-9}}}{{s^3 + 0.02028s^2 + 5.556 \times 10^{-6}s}} \\[1em]
					\dfrac{{0.0336s^2 + 2.987 \times 10^{-5}s + 5.704 \times 10^{-9}}}{{s^3 + 0.02028s^2 + 5.556 \times 10^{-6}s}} & \dfrac{{0.0336s^2 + 2.987 \times 10^{-5}s + 5.704 \times 10^{-9}}}{{s^3 + 0.02028s^2 + 5.556 \times 10^{-6}s}} \end{bmatrix}
			\end{equation}
			
			\clearpage
			
			Na slici (Sl. \ref{fig:sigmaINV}) su prikazane ostvarene singularne karakteristike funkcije osetljivosti i komlementarne osetljivosti regulisanog sistema, dok su na slici (Sl. \ref{fig:sigmaG_W_INV}) prikazane singularne karakteristike originalnog i regulisanog sistema. Za prikaz karakteristika su kor"s\'cene normalizovane matrice.
			
			\hspace{2cm}
			
			\begin{figure}[h]
				\centering
				\begin{subfigure}{0.6\linewidth}
					\centering
					\includegraphics[width=\linewidth]{slike/zad2/sigma_KG_tf_norm_inv.eps}
					\caption{Karakteristike osetljivosti i komp. osetljivosti}
					\label{fig:sigmaINV}
				\end{subfigure}
				\hfill
				\begin{subfigure}{0.6\linewidth}
					\centering
					\includegraphics[width=\linewidth]{slike/zad2/Sigma_G_W_inv.eps}
					\caption{Originalni sistem i funkcija povratnog prenosa}
					\label{fig:sigmaG_W_INV}
				\end{subfigure}
				\caption{Singularne karakteristike, upravljanje na bazi inverzije dinamike}
			\end{figure}
			
			\hspace{2cm}
			
			Sa slike (Sl. \ref{fig:sigmaINV}) vidimo da se gornja i donja singularna karakteristika osetljivosti spojene u celom opsegu u"cestanosti "sto je u skladu sa o"cekivanjima. Na osnovu funkcije osetljivosti možemo odrediti ostvareni propusni opseg učestanosti i on iznosi $\omega_0 = 0.0018 \frac{\text{rad}}{\text{s}}$. Za vršne vrednosti osetljivosti i komplementarne osetljivosti možemo reći da su oko preporučenih vrednosti, $M_S$ = 1.0711, $M_T$ = 1.000. Kao "sto vidimo na (Sl. \ref{fig:sigmaG_W_INV}) gornja i donja singularna karakteristika funkcije spregnutog prenosa su spojene na celom opsegu učestanosti.
			
			\clearpage
			
			
			
			\begin{figure}[!ht]
				\centering
				\includegraphics[width=0.7\linewidth]{slike/zad2/step_T_inv.eps}
				\caption{Zavisnost izlaza od vremena na jediničnu step pobudu}
				\label{fig:stepT_inv}
			\end{figure}
			
			\begin{figure}[!ht]
				\centering
				\includegraphics[width=0.7\linewidth]{slike/zad2/step_KS_inv.eps}
				\caption{Zavisnost upravljanja od vremena na jediničnu step pobudu}
				\label{fig:stepKS_inv}
			\end{figure}
			
			Na slici (Sl. \ref{fig:stepT_inv}) prikazani su odzivi na step pobudu. Dijagonalni grafici su zadovoljavajući, vandijagonalni grafici gpokazuju da je interakcija me\dj{}u kanalima u potpunosti potisnuta. Na slici (Sl. \ref{fig:stepKS_inv}) mo"zemo videti step odziv upravljanja, u pitanju su upravljanja prihvatljivih amplituda.
			
			\newpage
			\subsection{H$_\infty$ kontroler}
			
			Pri sintezi H$_\infty$ kontrolera polazi se od standardne P-K inerkonekcije prikazane na slici (Sl. \ref{fig:p_k_inter}):
			
			\begin{figure}[!h]
				\centering
				\includegraphics[width=0.5\linewidth]{slike/zad2/p_k_inter.png}
				\caption{Standardna postavka problema H$_\infty$ optimalnog upravljanja }
				\label{fig:p_k_inter}
			\end{figure}
			
			Vektor \textbf{v} je skup merenja, koja koristi kontroler \textbf{K}(s) pri generisanju upravljačkog vektora \textbf{u}. Signali eksternih ulaza (tipično referenca, poremećaji, šum merenja) predstavljeni su vektorom \textbf{w}, dok signali, vezani za evaluaciju performansi, čine komponente vektora \textbf{z}. \\
			
			U generalisani objekat \textbf{P}(s) uključene su matrica funkcija prenosa nominalnog objekta upravljanja i težinske funkcije za penalizaciju performansi. \\
			
			Zadatak problema \textbf{H}$_\infty$ - optimalnog upravljanja je pronalaženje, u skupu svih
			pravilnih stabilnih kontrolera \textbf{K}, onog optimalnog kontrolera $\textbf{K}_{\textbf{H}\infty}$, koji
			minimizira H$_\infty$ normu matrice funkcija spregnutog prenosa \textbf{N}, odnosno
			
			\begin{equation}
				\textbf{K}_{\textbf{H}_\infty} (s): \left\| \textbf{N}(\textbf{P},\textbf{K}_{\textbf{H}_\infty}) \right\|_\infty = \min_\textbf{K} \left\| \textbf{N}(\textbf{P},\textbf{K}) \right\|_\infty
			\end{equation}
			
			\noindent Ovaj problem još nije rešen, te se prelazi na rešavanje suboptimalnog problema. Suboptimalni problem se re"sava iterativno za neko $\gamma > \gamma_{opt}$ sve dok algoritam uspe"sno konvergira nastavlja se procedura polovljenja intervala. Kao finalni kontroler se usvaja onaj za koji se postignuta najmanja vrednost optimizactionog kriterijum.
			
			\clearpage
			\subsubsection{\textit{Mixed - sensivity} H$_\infty$ \textit{S/KS} kontroler }
			
			Kao što mu samo ime kaže, pored optimizacije funkcije osetljivosti, potrebno je minimizirati H$_\infty$ normu još neke druge matrice funkcija prenosa zatvorene sprege.\\
			
			\textit{S/KS} optimizacija se svodi na problem praćenja reference koji se može formulisati kao problem optimizacije ponderisane funkcije osetljivosti $\textbf{W}_1$(s)$\textbf{K}_1$(s), tj. kao
			nalaženje H$_\infty$-optimalnog kontrolera koji realizuje:
			
			\begin{equation}
				\min_\textbf{K} \|\textbf{W}_1(s)\textbf{S}(s)\|_\infty
				\label{eq:mix}
			\end{equation}
			
			\noindent Sa druge strane, mora se kontrolisati i angažovanje upravljanja u realizaciji
			prethodno postavljenog zadatka praćenja referentne vrednosti, jer
			optimizacija \eqref{eq:mix} može rezultovati kontrolerom, koji postavljeni zadatak rešava primenom neprihvatljivo velikih amplituda upravljanja. Stoga je, osim ovakve
			optimizacije, od interesa da i prenos od reference $r$ do upravljanja $u$ bude “dovoljno mali”, što se može formulisati kao nalaženje kontrolera koji obezbeđuje:
			
			\begin{equation}
				\min_\textbf{K} \|\textbf{W}_2(s)\textbf{K}(s)\textbf{S}(s)\|_\infty, 
				\label{eq:mix2}
			\end{equation}
			gde težinska funkcija $\textbf{W}_2$(s) egzaktno iskazuje zahtev “dovoljno mali” prenos
			u angažovanju upravljačkog napora pri praćenju reference. \\
			
			Kako bismo rešili dati problem, potrebno je najpre definisti P-K interkonekciju. Kako u vektor \textbf{z} treba da stoje veličine koje su nam od interesa za praćenje performansi sistema, vektor \textbf{z} sadrži ponderisanu funkciju osetljivosti i ponderisano angažovanje upravljačkog napora. Vektor \textbf{w} treba da bude željena referenca, te je \textbf{w} = r, što je prikazano na slici (Sl. \ref{fig:pk2}) :
			
			\begin{figure}[!h]
				\centering
				\begin{subfigure}[b]{0.45\textwidth}
					\includegraphics[width=\linewidth]{slike/zad2/p_k_H1.png}
					\caption{  }
					\label{fig:pk1}
				\end{subfigure}
				\hfill
				\begin{subfigure}[b]{0.45\textwidth}
					\includegraphics[width=\linewidth]{slike/zad2/p_k_H2.png}
					\caption{}
					\label{fig:pk2}
				\end{subfigure}
				\hfill
				\caption{  \textit{S/KS mixed-sensitivity} postavka u problemu praćenja reference:
					a) sistem zatvorene sprege upotpunjen težinskim funkcijama,
					b) formalna postavka $H_\infty$ optimizacije}
				
			\end{figure}
			
			U slučaju da se zadate performanse ostvarive, za ovako projektovanu P-K interkonekciju, primenom $H_\infty$ postupka projektovanja dolazi se do željenog kontrolera. Treba napomenuti da o"cekuje lo"se performansu po pitanju potiskivanja poreme\'caja zato "sto smo formulisali problem kao problem pra\'cenja reference.
			
			\clearpage
			
			\subsubsection{Projektovanje kontrolera}
			
			Najpre je potrebno odrediti težinske funkcije za $\textbf{S}$(s) i $\textbf{K}$(s)$\textbf{S}$(s). Cilj je da gornja i donja singularna karakteristika budu spojene, pa se prema tome definiše težinska funkcija: 
			
			\begin{equation}
				w(s) = g_s(s)^{-1}, \quad \textbf{W}(s) = w(s)\textbf{I} \implies (\forall \omega) \quad \overline{\sigma } (\textbf{W}(j\omega))= \underline{\sigma}(\textbf{W}(j\omega)) = |w(j\omega)|
			\end{equation}
			
			$g_s$(s) je oblikujuća funkcija i definiše specifikacije koje želimo da ostavrimo. \\
			
			Za funkciju osetljivosti $S_s$(s), jedan od mogćih rešenja za oblikujuću funkciju je:
			
			\begin{equation}
				g_s(s) = \frac{s + \omega_{WC} S_{0WC}}{\frac{s}{M_{SWC}} + \omega_{0WC}}
			\end{equation}
			\vspace{1cm}
			
			Parametri oblikujuće funckije su usvojeni tako da budu zadovoljimi sledeći zahteve za gornju singularnu karakteristiku osetljivosti $\overline{\sigma}(\textbf{S}(j\omega))$
			: \\
			
			- na niskim učestanostima bude vrednosti manjih od $S_{0WC}$, \\
			
			- na visokim učestanostima bude vrednosti manjih od $M_{0WC}$, \\
			
			- ostvaruje propusni opseg veći ili jednak od $\omega_{0WC}$. \\
			
			Na osnovu definisanih zahteva, usvojili smo sledeće vrednosti:
			
			\begin{equation}
				S_{0WC} = 10^{-4},~ \omega_{0WC} = 2.4 \times 10^{-2} \: \text{rad/s} \: \text{ i } \: M_{SWC} = 1.05 
			\end{equation}
			\vspace{1cm}
			
			Oblikujuća funckija za funckiju prenosa angažovanja upravljačkog napora pri praćenju reference ima jednostvaniji oblik i definiše samo maksimalnu vrednost pojačanja koja je prihvatljiva za naš sistem:
			
			\begin{equation}
				g_u(s) = K_u,
			\end{equation}
			
			\noindent gde je $K_u$ maksimalno dozvoljeno pojačanje funkcije $K_s$(s)$S_s$(s). Analizom našeg sistema, usvojeno je da je $K_u$ = 1.8. \\
			
			\vspace{1cm}
			Na kraju, dobijamo kontroler :
			
			
			\begin{equation}
				\begin{split}
					\hat{\textbf{K}}(s) &= \frac{1}{s^4 + 0.1392 s^3 + 0.003995 s^2 + 3.172 \times 10^{-10} s + 6.295 \times 10^{-18}} \\
					&= \begin{bmatrix}
						\phantom{-}8450 s^3 + 635.3 s^2 + 0.2387 s + 9.525 \times 10^{-9}  & \phantom{-}4565 s^3 + 618.6 s^2 + 0.3098 s + 1.242 \times 10^{-8} \\
						0.0004638 s^3 + 8.891 \times 10^{-5} s^2 + 5.5 \times 10^{-8} s & \phantom{-}0.002378 s^3 + 0.0002012 s^2 + 1.224 \times 10^{-7} s 
					\end{bmatrix}
				\end{split}
			\end{equation}
			
			\clearpage
			
			Na slici (Sl. \ref{fig:sigma_KG_inf}) su prikazane ostvarene singularne karakteristike funkcije osetljivosti i komlementarne osetljivosti regulisanog sistema, dok su na slici (Sl. \ref{fig:WS_WKS}) ponderisane karakteristike koje su dobijene procesom optimizacije. Mo"zemo primetiti da algoritam nije uspeo da konvergira, ali je dobije re"senje prihvatljivo.
			
			\hspace{2cm}
			
			\begin{figure}[h]
				\centering
				\begin{subfigure}{0.59\linewidth}
					\centering
					\includegraphics[width=\linewidth]{slike/zad2/WS_WKS.eps}
					\caption{Ponderisana osetljivost i upravljane}
					\label{fig:WS_WKS}
				\end{subfigure}
				\hfill
				\begin{subfigure}{0.59\linewidth}
					\centering
					\includegraphics[width=\linewidth]{slike/zad2/sigma_KG_inf.eps}
					\caption{Originalni sistem i funkcija povratnog prenosa}
					\label{fig:sigma_KG_inf}
				\end{subfigure}
				\caption{Singularne karakteristike, upravljanje na bazi inverzije dinamike}
			\end{figure}
			
			\hspace{2cm}
			
			Sa slike (Sl. \ref{fig:sigma_KG_inf}) vidimo da se gornja i donja singularna karakteristika osetljivosti spojene u celom opsegu u"cestanosti "sto je u skladu sa o"cekivanjima. Na osnovu funkcije osetljivosti možemo odrediti ostvareni propusni opseg učestanosti i on iznosi $\omega_0 = 0.0020 \frac{\text{rad}}{\text{s}}$. Za vršne vrednosti osetljivosti i komplementarne osetljivosti možemo reći da su oko preporučenih vrednosti, $M_S$ = 1.0290, $M_T$ = 1.000. 
			
			\clearpage
			
			
			
			\begin{figure}[!ht]
				\centering
				\includegraphics[width=0.7\linewidth]{slike/zad2/step_T_inf.eps}
				\caption{Zavisnost izlaza od vremena na jediničnu step pobudu}
				\label{fig:stepT_inf}
			\end{figure}
			
			\begin{figure}[!ht]
				\centering
				\includegraphics[width=0.7\linewidth]{slike/zad2/step_KS_inf.eps}
				\caption{Zavisnost upravljanja od vremena na jediničnu step pobudu}
				\label{fig:stepKS_inf}
			\end{figure}
			
			Na slici (Sl. \ref{fig:stepT_inf}) prikazani su odzivi na step pobudu. Dijagonalni grafici su zadovoljavajući, vandijagonalni grafici pokazuju da je interakcija me\dj{}u kanalima skoro u potpunosti potisnuta. Na lici (Sl. \ref{fig:stepKS_inf}) mo"zemo videti step odziv upravljanja, u pitanju su upravljanja prihvatljivih amplituda.
			
			
			
			
			\newpage
			
			\section{Komparativna analiza projektovanih sistema upravljanja}
			
			\subsection{Poređenje odziva sistema}
			\subsubsection{Odskočna promena reference na prvom kanalu}
			\begin{figure}[!ht]
				\centering
				\includegraphics[width= \linewidth]{slike/zad2/step_xa.eps}
				\caption{Odskočna promena reference na prvom kanalu}
				\label{fig:step_xa}
			\end{figure}
			
			Na slici (Sl. \ref{fig:step_xa}) mo"zemo videti odziv sistema na step promenu reference na prvom kanalu, u pitanju je step promena u iznosu 60\% maksimalne amplitude. Mo"zemo primetiti slede\'ce:
			\begin{itemize}
				\item Decentralizovano: Postoji izra"zena inverzija odziva na drugom kanalu usled neadekvatnog upravljanja bazom. Ovo je o"cekivano zbog dijagonalne prirode kontrolera.
				
				\item Dekupluju\'ce na nultoj u"cestanosti: I dalje postoji izra"zena reakcija na drugom kanalu, ali ovog puta je razlog previ"se agresivno upravljanje bazom, pa dolazi do preskoka u pH vrednosti. Dakle poku"saj kompenzacije postoji, ali nije adekvatan.
				
				\item Dekupluju\'ce na $\omega_0$: Postoji blaga inverzija odziva, me\dj{}utim vreme smirenja je pribli"zno isto kao u prethodna dva slu"caja. Efektivno bolje smanjujemo poja"canje, ali ne uti"cemo dovoljno na vremenske konstante.
				
				\item Inverzija dinamike: Blaga inverzija odziva, ali uspevamo da smanjimo i amplitudu i vremenske konstante u odzivu.
				
				\item $H_{\infty}$: Interakcija me\dj{}u kanalima prak"cno ne postoji u ovom slu"caju.
				
			\end{itemize}
			\clearpage
			\subsubsection{Odskočna promena reference na drugom kanalu}
			\begin{figure}[!ht]
				\centering
				\includegraphics[width = \linewidth]{slike/zad2/step_y.eps}
				\caption{Odskočna promena reference na drugom kanalu}
				\label{fig:step_y}
			\end{figure}
			\vspace{1cm}
			Na slici (Sl. \ref{fig:step_y}) mo"zemo videti odziv sistema na step promenu reference na drugom kanalu, u pitanju je step promena u iznosu 30\% maksimalne amplitude. Mo"zemo primetiti slede\'ce:
			\begin{itemize}
				\item Decentralizovano: Postoji izra"zena inverzija odziva na prvom kanalu usled neadekvatnog upravljanja kiselinom. Ovo je o"cekivano zbog dijagonalne prirode kontrolera. Vreme smirenja je i do dva puta du"ze nego u prethodnom slu"caju.
				
				\item Dekupluju\'ce na nultoj u"cestanosti: Interakcija me\dj{}u kanalima prak"cno ne postoji u ovom slu"caju.
				
				\item Dekupluju\'ce na $\omega_0$: Interakcija me\dj{}u kanalima prak"cno ne postoji u ovom slu"caju.
				
				\item Inverzija dinamike: Blaga inverzija odziva, ali uspevamo da smanjimo i amplitudu i vremenske konstante u odzivu.
				
				\item $H_{\infty}$: Interakcija me\dj{}u kanalima prak"cno ne postoji u ovom slu"caju.
				
			\end{itemize}
			
			
			\clearpage
			\subsubsection{Odskočni poremećaj na ulazu prvog kanala}
			
			\begin{figure}[!ht]
				\centering
				\includegraphics[width = \linewidth]{slike/zad2/d_Fa.eps}
				\caption{Odskočni poremećaj na ulazu prvog kanala}
				\label{fig:d_Fa}
			\end{figure}
			\vspace{1cm}
			Na slici (Sl. \ref{fig:d_Fa}) mo"zemo videti odziv sistema na step poreme\'caj na upravljanju prvog kanala, u pitanju je step promena u iznosu 10\% maksimalne amplitude. Mo"zemo primetiti slede\'ce:
			\begin{itemize}
				\item Decentralizovano: Poreme\'caj se potiskuje relativno brzo, ali postoji zna"cajan uticaj na koncetraciju baze.
				
				\item Dekupluju\'ce na nultoj u"cestanosti: Poreme\'caj se potiskuje relativno brzo, ali je izra"zeno dejstvo na amplitude pH vrednosti i baze.
				
				\item Dekupluju\'ce na $\omega_0$: Potiskuje poreme\'caj podjednako dobro kao i kontroler na bazi decentralizovanog upravljanja, ali uz manje odra"zavanje na koncentraciju kiseline.
				
				\item Inverzija dinamike: Lo"se potiskuje poreme\'caj, "sto je o"cekivano zato "sto je nije projektovan kontroler za potiksivanje poreme\'caja na bazi inverzije dinamike.
				
				\item $H_{\infty}$: Lo"se potiskuje poreme\'caj, "sto je o"cekivano zato "sto je nije projektovan kontroler na osnovu SGd/KSGd optimizacije.
				
			\end{itemize}
			
			\clearpage
			
			\subsubsection{Odskočni poremećaj na ulazu drugog kanala}
			
			\begin{figure}[!ht]
				\centering
				\includegraphics[width= \linewidth]{slike/zad2/d_Fb.eps}
				\caption{Odskočni poremećaj na ulazu drugog kanala}
				\label{fig:d_Fb}
			\end{figure}
			
			Na slici (Sl. \ref{fig:d_Fb}) mo"zemo videti odziv sistema na step poreme\'caj na upravljanju drugog kanala, u pitanju je step promena u iznosu 10\% maksimalne amplitude. Mo"zemo primetiti slede\'ce:
			\begin{itemize}
				\item Decentralizovano: Poreme\'caj se potiskuje relativno brzo, ali postoji zna"cajan uticaj na koncetraciju kiseline.
				
				\item Dekupluju\'ce na nultoj u"cestanosti: Poreme\'caj se potiskuje relativno brzo, ali je izra"eno dejstvo na amplitude pH vrednosti i baze.
				
				\item Dekupluju\'ce na $\omega_0$: Potiskuje poreme\'caj bolje nego kontroler na bazi decentralizovanog upravljanja i to uz manje odra"zavanje na koncentraciju kiseline i baze.
				
				\item Inverzija dinamike: Lo"se potiskuje poreme\'caj, "sto je o"cekivano zato "sto je nije projektovan kontroler za potiksivanje poreme\'caja na bazi inverzije dinamike.
				
				\item $H_{\infty}$: Lo"se potiskuje poreme\'caj, "sto je o"cekivano zato "sto je nije projektovan kontroler na osnovu SGd/KSGd optimizacije.
				
			\end{itemize}
			
			\newpage
			\subsubsection{Odziv na direkcionalnost pobude sa izra"zenom multivarijabilno"\'cu}
			
			\begin{figure}[!ht]
				\centering
				\includegraphics[width= \linewidth]{slike/zad2/dir.eps}
				\caption{Odskočni poremećaj na ulazu drugog kanala}
				\label{fig:dir}
			\end{figure}
			\vspace{1cm}
			Na slici (Sl. \ref{fig:dir}) mo"zemo videti odziv sistema na pobudu "cija direkcionalnost ima najizra"zeniju multivarijabilnost u originalnom sistemu (Sl. \ref{fig:direkc_2}), vektor pobude je jedini"cne du"zine. Mo"zemo primetit slede\'ce: 
			\begin{itemize}
				\item Decentralizovano: Kao "sto je i o"cekivano ovaj kontroler ima pove\'canu osetljivost na direkcionalnost pobude zato "sto je inherentno monovariabilan.
				
				\item Dekupluju\'ce na nultoj u"cestanosti: Direkcionalnost pobude ima blag uticaj na odziv, ali se oblik odziva ne menja.
				
				\item Dekupluju\'ce na $\omega_0$: Direkcionalnost pobude ima blag uticaj na odziv, ali se oblik odziva ne menja.
				
				\item Inverzija dinamike: Direkcionalnost pobude nema uticaja na odziv.
				
				\item $H_{\infty}$: Direkcionalnost pobude nema uticaja na odziv.
			\end{itemize}
			
			
			
			\clearpage
			
			
			\subsection{Komparativni dijagrami singularnih karakteristika}
			
			Na (Sl. \ref{fig:1dof}) prikazan je sistem upravljanja u povratnoj sprezi za (multivarijabilni) objekat upravljanja G(s), pomoću (multivarijabilnog) rednog kontrolera K(s) jednog stepena slobode. Uticaj poremećaja predstavljen je
			(multivarijabilnim) modelom poremećaja$G_d$(s), a označeni su eksterni vektori ulaznih signala: referenca r, poremećaj d i šum merenja n.
			\begin{figure}[!h]
				\centering
				\includegraphics[width=0.7\linewidth]{slike/zad2/1dof.jpg}
				\caption{Sistem sa kontrolerom jednog stepena slobode u povratnoj sprezi}
				\label{fig:1dof}
			\end{figure}
			
			\noindent Cilj upravljanje jeste da se ostvari idealno praćenje zadate reference, odnosno ostavrenje nulte greške \textit{e = r - y}, sposobnost sistema da se bori protiv šuma i otkloni sve vrste poremećaja. 
			\vspace{2cm}
			
			Kako bismo objedinili tražene zahteve, definisaćemo funkciju osetljivosti S(s) (prenos od reference do greške praćenja) i funkciju komplementarne osetljivosti T(s) (prenos od reference do izlaza sistema). Kako im samo ime kaže, važi sledeća relacija: 
			
			\begin{equation}
				S + T = I
			\end{equation}
			
			\noindent Koristeći kombinovano pravilo za interkonekcije, sa slike (Sl. \ref{fig:1dof}) , možemo napisati sledeće jednačine:
			
			\begin{equation}
				\begin{split}
					y &= T(s)r - T(s)n + S(s)G_d(s)d \\
					e &= S(s)r + T(s)n - S(s)G_d(s)d \\
					u &= K(s)S(s)r - K(s)S(s)n - K(s)S(s)G_d(s)d
				\end{split}
			\end{equation}
			
			\vspace{2cm}
			Na osnovu jednačina možemo zaključiti da ne mogu biti ispunjeni svi uslovi na celom opsegu učestanosti. Odnosno, možemo preformulisati tražene zahteve na nekom od opsega učestanosti od interesa : \\
			
			- za praćenje referentne vrednosti: $(\omega \leq \omega_r)$ $\sigma(S(j\omega)) \ll 1$ $\sigma(T(j\omega)) \approx 1$, \\
			
			- za potiskivanje poremećaja : $(\omega \leq \omega_d)$ $\sigma(S(j\omega) G_d(j\omega)) \ll 1$,  \\
			
			- za potiskivanje šuma: $(\omega \geq \omega_n)$ $\sigma(T(j\omega)) \ll 1$, \\
			
			gde su $\omega_r, \omega_d, \omega_n$ granice frekvencijskog sadržaja \textit{r,d,n} eksternih ulaznih signala.
			
			
			
			\newpage
			
			\subsubsection{Matrica funkcije osetljivosti S(s)}
			Funkcija osetljivosti predstavlja prenos od reference do greške praćenja reference, koja se definiše kao: 
			
			\begin{equation}
				S(s) = (I + G(s)K(s))^{-1}
			\end{equation}
			
			\begin{figure}[!ht]
				\centering
				\includegraphics[width=0.9\linewidth]{slike/zad2/Sigma_S_komp.eps}
				\caption{Singularne karakteristike matrica funkcija osetljivosti sistema upravljanja}
				\label{fig:sigmaS_komp}
			\end{figure}
			
			Na slici (Sl. \ref{fig:sigmaS_komp}) prikazan je uporedni dijagram gornjih i donjih singularnih karakteristika matrice funckije osetljivosti za sve projektovane sisteme upravljanja. Očigledno je da decentralizovano upravljanje daje najlošije rezultate. Najbolje rezultate postižemo pri projektovanju kontrolera na bazi inverzije dinamike, korišćenjem dinamičkog dekuplera. Efekat dekuplovanje se proteže kroz ceo opseg učestanosti. Posle njega, najbolje rezultate daje H$_\infty$ kontroler. Dekupler na učestanosti propusnog opsega daje očekivane rezultate, odnosno, dekuplovanje je uspešno tek od učestanosti propusnog opsega. Odnosno, možemo zaključiti da jedino decentralizovano upravljanje zaostaje od svih prikazanih medota. 
			
			\newpage
			\subsubsection{Matrica funkcije komplementarne osetljivosti T(s)}
			
			Funkcija komplementarne osetljivosti predstavlja prenos od reference do izlaza, koja se definiše kao: 
			
			\begin{equation}
				T(s) = G(s)K(s)(I + G(s)K(s))^{-1}
			\end{equation}
			
			
			
			\begin{figure}[!ht]
				\centering
				\includegraphics[width=0.9\linewidth]{slike/zad2/Sigma_T_komp.eps}
				\caption{Singularne karakteristike matrica funkcija komplementarne osetljivosti sistema upravljanja}
				\label{fig:sigmaT_komp}
			\end{figure}
			Na slici (Sl. \ref{fig:sigmaT_komp}) prikazan je uporedni dijagram gornjih i donjih singularnih karakteristika matrice funckije komplementarne osetljivosti za sve projektovane sisteme upravljanja. Možemo primetiti da svi zaključci do kojih smo došli analizirajući karakteristike funkcije osetljivosti važe i ovde, te ih nećemo ponavljati. 
			
			\newpage
			
			\subsubsection{Procena propusnog opsega i maksimalnih vrednosti funkcije osetljivosti i komplementarne osetljivosti}
			
			
			Ostalo je još da prokomentarišemo ostvarene maksimume funkcije osetljivosti i komplementarne osetljivosti i ostvarene propusne opsege. Što se tiče vršnih vrednosti $M_S$ i $M_T$ možemo zaključiti da svi projektovani sistemi upravljanja daju slične rezultate. Sa $H_\infty$ ostvarujemo najveći i željeni propusni opseg, dok sa  decentralizovanim upravljanjem dobijamo najmanji propusni opseg, za jedan red veli"cine manji od željenog. Ostali sistemi ostvaruju zadovoljavajuće propusne opsege. 
			
			\begin{table}[h]
				\centering
				\begin{tabular}{|c|c|c|c|c|c|}
					\hline
					& $Dec$  & $Dek_0$  & $Dek_{w0}$ & $Inv$ & $H_\infty$\\ \hline
					$M_S$ & 1.0010   & 1.0543 & 1.0000   & 1.0711 & 1.0290  \\ \hline
					$M_T$ & 1.0313   & 1.0888 & 1.0477   & 1.0000 & 1.0000  \\ \hline
				\end{tabular}
				\caption{Maksimalne vrednosti funckije osetljivosti i kompementarne osetljivosti}
				\label{tab:msmt}
			\end{table}
			
			
			
			\begin{table}[h]
				\centering
				\begin{tabular}{|c|c|c|c|c|c|}
					\hline
					& $Dec$  & $Dek_0$  & $Dek_{w0}$ & $Inv$ & $H_\infty$\\ \hline
					$\omega_0$ & 6.3148*$10^{-4}$   & 0.0013
					& 0.0017   & 0.0018 & 0.0020  \\ \hline
				\end{tabular}
				\caption{Ostvareni propusni opsezi}
				\label{tab:propusni_opseg}
			\end{table}
			
			\clearpage
			\section{Zaklju"cak}
			\vspace{0.5cm}
			
			Kroz sam rad upoznali smo se sa osnovama multivarijabilih sistema, konkretno na CSTR sistemu. Naš zadatak bio je da održavamo PH vrednost i koncentraciju kiseline tečnosti u rezervoaru konstantnim. Na osnovu jednačina koje opisuju fizičke procese u sistemu dobijen je MIMO model sistema. Kako dati sistem ima beskonačno ravnotežnih stanja, razmatrajući literaturu sa interneta, odabrali smo jedan radni režim. Nakon toga izvršena je linearizacija oko izabranong nominalnog režima i uporedno su prikazani grafici izlaza u slučaju originalnog nelinearnog i linearizovanog modela oko nominalnog režima. Određene su vremenske konstane prelaznih procesa na osnovu kojih možemo doći do zaključka da je promena koncentracije kiseline u rezervoaru nešto brža do PH vrednosti. \\
			
			Razmotrili smo potencijalne poremećaje koji mogu uticati na rad samog sistema i da sve možemo modelovati kao poremećaj na izlazu, konkrento kao promenu PH vrednosti. \\
			
			Određivanjem polova i nula sistema, zaključujemo da je sistem stabilan i da nema transmisionih nula. RGA analizom(frekvencijski zavisna RGA matrica i RGA broj) dolazimo do zaključka da je dijagonalno uparivanje bolje od vandijagonalnog (RGA na nultoj učestanosti pokazuje negativne vandijagonalne elemenete). \\
			
			Uveli smo skaliranje objekta opravljanja prilikom procesa upravljanja usled velikih razlika u nominalni vrednostima izlaza, u pitanju je razlika reda veli"cine $10^5$. Nakon normalizaicije mo"zemo na uniforman na"cin da vr"simo analizu kvaliteta odziva prvog, odnosno drugog, izlaza sistema.\\
			
			U drugom delu projektnog zadatka bavili smo se projektovanjem kontrolera. Cilj nam je bio da isprojektujemo kontrolere kojima ćemo ostvariti željene performanse(propusni opseg i pretek faze) i stabilnost. Kako bi projektovanje kontrolera bilo što jednostavnije moguće, ideja je bila raspregnuti kanale što je više moguće. Razmotrili smo pet različitih metoda projektovanja za MIMO sisteme. \\
			
			Prvi kontroler kojim smo se bavili bio je decentralizovani kontroler. On je ujedno i najednostavniji kontroler koji smo isprojektovali i osnovu za njegovo projektovanje nam je dala RGA analiza koja je pokazala da je dijagonalno uparivanje bolje. Pored jednostavnosti ovog kontrolera, njegova najveća prednost je ta što finalni kontroler zapravo nije multivarijabilan. Međutim, sistem je suštinski ostao spregnut što vidimo na osnovu vremenskih odziva i ostvarenog propusnog opsega koji je 100 puta manji od željenog, što je ujedno i njegova najveća mana. \\
			
			Naredna dva kontrolera su nešto složenija, pošto se sastoje od statičkog dekuplera i decentralizovanog kontrolera. Dekuplovanja su vršena na nultoj učestanosti(dekupler ustaljenog stanja) i na učestanosti propusnog opsega(dekupler na učestanosti propusnog opsega). Kao rezultat dobijamo bolje rasprezanje sistema, s tim da bolje rezulatate dobijamo sa dekuplerom na učestanosti propusnog opsega. Mana je, što se efekat dekuplovanja ostvaruje samo na nekom od opsega učestanosti. \\
			
			Kako bismo dobili dekuplovanje na celom opsegu učestanosti, koriščen je složenije dekupler, tačnije dinamički dekupler. U radu smo se odlučili da isprojektujemo kontroler na bazi potpune inverzije dinamike. Sa ovim kontrolerom dobijamo bolje rezultate po pitanju ostavrenog propusnog opsega, brzine i efekta dekuplovanja, ali mana je taj što se dobija složeniji kontroler. \\
			
			I kao poslednji kontroler, projektovan je $H_{\infty}$, koji pored traženih projektnih zahteva minimizuje kriterijumsku funkciju. Ovaj kontroler je najsloženiji od svih pomenutih, ali ostvaruje najbolje performanse po pitanju traženih zahteva. \\
			
			\clearpage
			
			U trećem delu izvršena je komparativna analiza svih projektovanih kontrolera. Komparativna analiza je izvršena na originalnom nelinearnom modelu. Razmatrani su odzivi pri odskočnim promenama reference prvog i drugog izlaza nezavisno. Analizirano je i ponašanje sistema u prisustvu poremećaja u vidu step promene koncentracije kiseline, odnosno baze.\\ 
			
			Na kraju su analizirane karakteristike sistema u frekvencijskom domenu, tačnije singularne karakteristike osetljivosti i komplementarne osetljivosti. Videli smo da decentralizovano upravljanje odskače u odnosu na ostale kontorlere obzirom da su mu donja i gornja singlurana karakteristika razmaknute na celom opsegu učestanosti, dok je $H_{\infty}$ dao najbolje rezultate. Vršne vrednosti $M_S$ i $M_T$ pri svim projektovanim kontrolerima su u dozvoljenim granicama. \\ 
			
			Projektovani kontroleri daju jako sli"cne performanse po svim kriterijumima. Zato ne mo"ze da se donese jednozna"can zaklju"cak koji bi kontroler bio bolji, mnogo razli"citih faktora uti"cu na izbog optimalnog kontrolera. Zbog toga prikaza\'cemo tabelarno osobine svih isprojektovanih kontrolera gde \'cemo oceniti performanse od najboljih do najgorih. Najvi"sa ocena je 1, a najni"za je 5.\\
			
			\vspace{0.5cm}
			
			
			
			\begin{table}[!h]
				\centering
				\begin{tabular}{|llllll|}
					\hline
					\multicolumn{6}{|c|}{Pore\dj{}enje kontrolera}                                                                                                                                                                                                                 \\ \hline
					\multicolumn{1}{|l|}{}              & \multicolumn{1}{l|}{slo"zenost} & \multicolumn{1}{l|}{pra\'cenje ref.} & \multicolumn{1}{l|}{potiskivanje porem.} &  \multicolumn{1}{l|}{multivarijabilnost} & prosek \\ \hline
					\multicolumn{1}{|l|}{$K_{dec}$}     & \multicolumn{1}{l|}{1}          & \multicolumn{1}{l|}{5}                              & \multicolumn{1}{l|}{3}                   &  \multicolumn{1}{l|}{5}    &  2.8   \\ \hline
					\multicolumn{1}{|l|}{$K_{dek0}$}    & \multicolumn{1}{l|}{2}          & \multicolumn{1}{l|}{4}                              & \multicolumn{1}{l|}{2}                   &  \multicolumn{1}{l|}{4}    & 2.4   \\ \hline
					\multicolumn{1}{|l|}{$K_{dek\omega_0}$} & \multicolumn{1}{l|}{3}          & \multicolumn{1}{l|}{3}                              & \multicolumn{1}{l|}{1}                   &  \multicolumn{1}{l|}{3}    & 2   \\ \hline
					\multicolumn{1}{|l|}{$K_{invF}$}  & \multicolumn{1}{l|}{4}          & \multicolumn{1}{l|}{2}                              & \multicolumn{1}{l|}{5}                   &  \multicolumn{1}{l|}{1}    & 2.4   \\ \hline
					\multicolumn{1}{|l|}{$K_{H_{\infty}}$}  & \multicolumn{1}{l|}{5}          & \multicolumn{1}{l|}{1}                              & \multicolumn{1}{l|}{4}                   &  \multicolumn{1}{l|}{2}    & 2.4   \\ \hline
					
				\end{tabular}
				\captionof{table}{}\label{poredjenje_kontrolera}
			\end{table}
			\vspace{1cm}
			
			
			Na osnovu (Tabela \ref{poredjenje_kontrolera}) vidimo da ako su svi kriterijumi podjednako bitni da je kontroler na bazi dekuplovanja na u"cestanosti propusnog opsega najbolji. Me\dj{}utim ako je mnogo ve\'ci akcenat na ekonomi"cnosti kao neki kompromis izme\dj{}u slo"zenosti postupka projektovanja i ostvarenih performansi se nam\'ce se kontroler na bazi decentralizovanog upravljanja, zato "sto se projektuje kao niz nezavisnih PI kontura, a performanse su u kontekstu ovog sistema prihvatljive. Ako je akcenat pra\'cenju reference onda je re"senje $H_{\infty}$ kontroler, njegova najve\'ca mana jeste manjak lako podesivih parametara, "sto nije slu"caj kod decentralizovanih kontrolera i kontrolera na bazi dekuplovanja, gde u su"stini imamo multivarijabilnu implementaciju PI kontrolera. Me\dj{}utim treba napomenuti da bi kontroleri na bazi inverzije dinamike i $H_{\infty}$ postigli zna"cajno bolje rezultate po pitanju potiskivanja poreme\'caja da su projektovani sa tom namenom na umu.\\
			
			Kao zaklju"cak mo"zemo da ka"zemo da sve do sad kori"s\'cene metode projektovanja rezultuju dobrim kontrolerima, glavna mana decentralizovanog upravljanja je odsustvo mehanizma za potiskivanje interakcija me\dj{}u petljama, dok je glavna mana metoda na bazi inverzije dinamike i optimalnog upravljanja to "sto zahtevaju jako dobro poznavanje modela. Kao dobar kompromis, izme\dj{}u ova dva pristupa, predstavlja dekupluju\'ce upravljanje, naime su"stinski ovo upravljanje predstavlja multivarijabilnu realizaciju PI kontrolera, pa bi bilo mogu\'ce 'podesiti' vrednosti matrice $W_1$ ru"cno na osnovu iskustva - zato "sto postoji jasna interpretacija dejstva rezultuju\'ceg kontrolera. Me\dj{}utim ni ovaj pristup nije bez mana, za razliku od $H_{\infty}$ nemamo garancije optimalnosti, niti jednostavnost projektovanja kao kod decentralizovanog upravljanja.
			
			
			\newpage
			
			\begin{thebibliography}{9}
				\bibitem{inicijalna}
				\emph{Robust Stable Nonlinear Control
					and Design of a CSTR in a Large
					Operating Range
				}, Johannes Gerhard, Martin M¨onnigmann,
				Wolfgang Marquardt
				\bibitem{}\emph{Nonlinear pH Control in a CSTR
				} RaynxId A. Wright al Costas Kravais
				\bibitem{beleske}
				\emph{https://automatika.etf.bg.ac.rs/sr/13e054msu}, bele"ske sa predavanja
				
				
				
				\bibitem{model}
				\emph{Dynamics of pH in Controlled Stirred Tank Reactor}, Thomas J. McAvoy,l Elmer HSU, and Stuart Lowenthal
				\bibitem{}\emph{Hybrid simulation of a pH stirred
					tank control system} Thomas J. McAvoy
				
				
			\end{thebibliography}
		\end{document}
