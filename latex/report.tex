\documentclass[a4paper,11pt]{article}
\usepackage{times}
%\setcounter{secnumdepth}{0} % sections are not getting numbered
\usepackage[english,serbian]{babel}


\usepackage[T1]{fontenc} 
\usepackage{biblatex} % bibliography

\addbibresource{citation.bib} % file with references


\usepackage{comment}
\usepackage{amsfonts}
\usepackage{amsmath}
\usepackage{amsthm}
\usepackage{IEEEtrantools}
\usepackage{graphicx}


%\usepackage{geometry}
%\usepackage{upgreek}
\usepackage[serbian]{babel}
%\usepackage{ulem}
%\usepackage{environ}
\usepackage{tikz}
\usepackage{color}
\usepackage{fancybox}

%\numberwithin{equation}{section}
\theoremstyle{definition} \newtheorem{deff}{Definicija}[section]
\theoremstyle{definition} \newtheorem{prim}[deff]{Primer}
\theoremstyle{plain} \newtheorem{teor}[deff]{Teorema}


\newcommand{\unija}[2]{#1 \cup #2}
\newcommand{\pres}[2]{#1 \cap #2}
\newcommand{\tnorm}{$t$-norm}
\newcommand{\tkonorm}{$t$-konorm}


\renewenvironment{proof}[1][\proofname]{{\bfseries #1.}}

\frenchspacing


\usepackage[a4paper,top=3cm,bottom=2cm,left=2cm,right=2cm,marginparwidth=1.75cm]{geometry}
%% Useful packages
\usepackage{mathrsfs}
\newsavebox\foobox
\newlength{\foodim}
\newcommand{\slantbox}[2][0]{\mbox{%
		\sbox{\foobox}{#2}%
		\foodim=#1\wd\foobox
		\hskip \wd\foobox
		\hskip -0.5\foodim
		\pdfsave
		\pdfsetmatrix{1 0 #1 1}%
		\llap{\usebox{\foobox}}%
		\pdfrestore
		\hskip 0.5\foodim
}}
\def\Laplace{\slantbox[-.45]{$\mathscr{L}$}}

%\usepackage[colorinlistoftodos]{todonotes}

\usepackage{caption}
\usepackage{subcaption}
\usepackage{changepage}


\usepackage{blindtext}

\usepackage{tabularx}
\usepackage[export]{adjustbox}

\usepackage[utf8]{inputenc}
\usepackage[T1]{fontenc}
\usepackage{lmodern}
\usepackage{graphicx}
\usepackage{color}
\usepackage{listings}
\usepackage{amsmath}

%\usepackage[usenames,dvipsnames]{xcolor}
%\usepackage[colorlinks=true,linkcolor=blue]{hyperref}

\usepackage{amsfonts}
\usepackage{epstopdf}

\usepackage{float}

\usepackage[shortlabels]{enumitem}
\usepackage[yyyymmdd]{datetime}


\renewcommand{\figurename}{Slika}

\DeclareMathOperator*{\argmax}{\arg\max}
\graphicspath{{./images/}}



\usepackage{siunitx}

\usepackage{scalerel}



\sloppy

\epstopdfsetup{update} % only regenerate pdf files when eps file is newer

%%%%%%%% DOCUMENT %%%%%%%%
\setlength {\marginparwidth }{2cm}
\begin{document}
	
	%%%% Title Page
	\begin{titlepage}
		
		\newcommand{\HRule}{\rule{\linewidth}{0.5mm}} 							% horizontal line and its thickness
		\center 
		
		% University
		\textsc{\LARGE Elektrotehnički fakultet u Beogradu}\\[1cm]
		
		% Document info
		\textsc{\Large Optimalno upravljanje sistemima}\\[0.2cm]
		\textsc{\large 13M051OUS}\\[1cm] 										
		\HRule \\[0.8cm]
		{ \huge \bfseries Upravljanje dvostrukim inverznim klatnom}\\[0.7cm]								% Assignment
		%\HRule \\[2cm]
		\textsc{\large Projektni zadatak broj 2}\\[1cm]
		
		
		\large
		\vfill 
		\emph{Studenti:}\\
		Nikita Jokić 3279/2023\\[0.1cm]
		Ivona Dučić 3067/2023\\[1.5cm]		
		\emph{Mentor:}\\
		doc. dr Aleksandra Krstić\\[0.1cm]									
		{\large Februar 2024}\\[2cm]
	\end{titlepage}
	\tableofcontents
	\newpage
	
	\section{Modeliranje sistema i analiza modela}
	\subsection{Uvod} 
	
	Furuta penduluma (FP) je rotaciono klatno koje se pokreće motorom jednosmerne struje. U ovom radu razmatrane su kontrolne strategije za nelinearni problem uspravljanja i balansiranja FP u njegovom nestabilnom, uspravnom ravnotežnom položaju. Iako ovo može delovati kao akademski problem, FP je ilustrativan za širok spektar dinamičkih sistema sa stvarnim primenama. Fokusiranje na ovu problematiku integrisano je u aktuelna istraživanja u oblasti sajber-fizičkih sistema (CPSs)\footnote{Sajber-fizički sistemi (CPSs) predstavljaju integraciju računarskih elemenata sa fizičkim sistemima, stvarajući tako dinamičko i uzajamno povezano okruženje.} \cite{inicijalna}. \\ 
	
	Fizičari su uspešni u modelovanju fenomena u prirodi u iznenađujućem broju redova veličina. Iako se u većini slučajeva može samo težiti posmatranju, u malom, ali važnom podskupu, moguće je aktivno modifikovati ponašanje sistema. Ovo se postiže uključivanjem računarskih komponenti koje interaguju sa fizičkim sistemom, što čini osnovu istraživanja u oblasti sajber-fizičkih sistema. Teorija upravljanja se bavi izazovima u postizanju željenih performansi sistema, uključujući nestabilnost u otvorenoj sprezi, ograničenja u broju promenljivih koje se mogu aktivirati i ograničene informacije o stanju sistema.\\
	
	Ovaj rad proučava sisteme koji se mogu modelovati konačnim brojem povezanih diferencijalnih jednačina prvog reda, kako linearne sisteme tako i nelinearne sistema, koji su prisutni u sve većem broju modernih aplikacija. \\
	
	Furuta pendulum, kao jednostavan primer nelinearnog sistema, služi kao osnovno sredstvo za istraživanje ideja u oblasti nelinearnog upravljanja. Takođe predstavlja prototip za praktično važne uređaje poput robotskih ruku cilindričnog oblika, rotacionih dizalica ili transportnih sistema za visoke objekte. Kontrolni problemi usmeravanja i balansiranja penduluma oko nestabilne ravnoteže su dobro proučeni.\\
	
	U rada \cite{inicijalna} se navodi da do trenutka kada je rad napisan, nelinearni problem uspravljanja FP nije rešavan optimalnom kontrolom. Korišćenjem optimalne kontrole, problem se elegantno formuliše kao minimizacija odgovarajućeg troška uz poštovanje određenih ograničenja. Primenom ove tehnike, autor istražuje savremene tehnike kontrole, njihovu primenljivost u sličnim problemima i razvija odgovarajuću kontrolnu tehniku za ovu klasu uređaja. Pored optimalne kontrole uspravljanja FP razmatrane su i ad hoc strategije koje se oslanjaju na zakone upravljanja izvedene iz analize sistema. Balansiranje klatna oko njegovog nestabilnog ravnotežnog stanja se obavlja korišćenjem LQG. 
	
	(uvod doraditi na kraju, navesti strukturu izvestaja...)
	
	
	\newpage
	
	\subsection{Modeliranje sistema} 
	
	Furuta pendulum je dinamički sistem koji se sastoji od dve osnovne komponente: \\
	
	Horizontalna ruka : Ovo je čvrsta šipka ili krak koji je postavljen horizontalno. Na jednom kraju može biti pričvršćen za oslonac, dok je drugi kraj često povezan sa vertikalnim delom sistema, poznatim kao pendulum. Horizontalna ruka služi kao platforma za postavljanje i rotiranje vertikalnog penduluma. \\
	
	- Vertikalni pendulum: Ovaj deo predstavlja masu (obično šipka ili krak) koje je povezano sa horizontalnom rukom na jednom kraju, a drugi kraj se slobodno kreće. Rotacija penduluma oko vertikalne ose može se kontrolisati pomoću motora koji je integrisan u sistem. Ovaj motor omogućava sistemu da realizuje oscilatorne pokrete.
	
	(DODAT SLIKU) i šta je štaa, da li definisati sve od konstanti što se pojavljuje u jednačinama
	\\
	
	Nelinearni model FP izveden je iz Langranžove mehanike. Energija sistema koji se sastoji od N krutih tela može se napisati kao zbir kinetičke energije $\tau$ i potencijalne energije $\nu$ \cite{inicijalna}. 
	\begin{equation}
		\tau = \sum_{i=1}^{N} \left( \frac{1}{2} M_i |v_i|^2 + \frac{1}{2} \omega_i I_i \omega_i^T \right)
	\end{equation}
	
	\begin{equation}
		\nu = \sum_{i=1}^{N} M_i \cdot (-g)
	\end{equation}
	
	
	
	Gde za svako kruto telo $i$, $M_i$ je masa tela, $v_i$ je brzina centra mase, $I_i$ matrica inercije, $\omega_i$ ugaona brzina, $r_i$ pozicija od centra mase, i $g$ je gravitaciono ubrzanje.\\
	
	Langranžijan se računa kao :
	
	\begin{equation}
		\mathcal{L} = \tau - \nu, 
	\end{equation}
	
	dok se jednačine kretanja računaju preko Ojler-Langranžovih jednačina: 
	\begin{equation}
		\frac{d}{dt} \left(\frac{\partial \mathcal{L}}{\partial \dot\alpha}\right) - \frac{\partial \mathcal{L}}{\partial \alpha} = -K_{a_1} \dot\alpha + K_f i, \quad 
	\end{equation}
	
	\begin{equation}
		\frac{d}{dt} \left( \frac{\partial \mathcal{L}}{\partial \dot\beta} \right) - \frac{\partial \mathcal{L}}{\partial \beta} = -K_{a_2} \dot\beta
	\end{equation}
	
	Karakteristika motora data je jednačinom: 
	\begin{equation}
		\frac{d}{dt} \left( L_b \frac{di}{dt} \right) + K_t \frac{d\alpha}{dt} + R i = u
	\end{equation}
	
	Kako bi se konstruisao nelinearni model sistema u prostoru stanja: 
	\begin{equation}
		\dot x = f(x, u)
	\end{equation}
	promenljive se definišu kao: $x_1$ = $\alpha$, $x_2$ = $\dot\alpha$,  $x_3$ = $\beta$, $x_4$ = $\dot\beta$, $x_5$ = $i$. \\
	
	
	Konačno model u prostoru stanja glasi:
	
	\begin{equation}
		\begin{aligned}
			\dot{x}_1 &= x_2; \\[0.8em]
			\dot{x}_2 &= -\frac{J_2(K_{a1}x_2 - K_fx_5 + x_4(L_{cm2}L_{e1}m_2x_4 + 2J_2x_2\cos(x_3))\sin(x_3))}{-L_{cm2}^2L_{e1}^2m_2^2\cos(x_3)^2 + J_2(J_0 + J_2\sin(x_3)^2)} \\[0.8em]
			&\quad + \frac{L_{cm2}L_{e1}m_2\cos(x_3)(-K_{a2}x_4 + (gL_{cm2}m_2 + J_2x_2^2\cos(x_3))\sin(x_3))}{-L_{cm2}^2L_{e1}^2m_2^2\cos(x_3)^2 + J_2(J_0 + J_2\sin(x_3)^2)}; \\[0.8em]
			\dot{x}_3 &= x_4; \\[0.5em]
			\dot{x}_4 &= \frac{K_{a2}x_4 - gL_{cm2}m_2\sin(x_3)(J_0 + J_2\sin(x_3)^2)}{L_{cm2}^2L_{e1}^2m_2^2\cos(x_3)^2 - J_2(J_0 + J_2\sin(x_3)^2)} \\[0.8em]
			&\quad + \frac{\cos(x_3)(-J_0J_2x_2^2 + L_{cm2}^2L_{e1}^2m_2^2x_4^2)\sin(x_3)}{L_{cm2}^2L_{e1}^2m_2^2\cos(x_3)^2 - J_2(J_0 + J_2\sin(x_3)^2)} \\[0.8em]
			&\quad - \frac{J_2^2x_2^2\sin(x_3)^3 + L_{cm2}L_{e1}m_2(K_{a1}x_2 - K_fx_5 + J_2x_2x_4\sin(2x_3))}{L_{cm2}^2L_{e1}^2m_2^2\cos(x_3)^2 - J_2(J_0 + J_2\sin(x_3)^2)};\\[0.5em]
			\dot{x}_5 &= -\frac{K_tx_2 - Rx_5 + u}{L_b}.
		\end{aligned}
		\label{eq:nonModel}
	\end{equation}
	
	
	\subsubsection{Matlab model}
	
	
	
	
	
	
	\clearpage
	
	
	\subsection{Ponašanje sistema u otvorenoj sprezi}
	
	
	
	\clearpage 
	\subsection{Linearizacija sistema}
	\label{sec:linearizacija}
	
	Nelinearni model (\ref{eq:nonModel}) se može linearizovati razvijanjem u Tejlorov red u okoline tačke $(\bar{x}, \bar{u})$:
	\begin{equation}
		\mathbf{f}(\mathbf{x}, u) \approx \mathbf{f}(\mathbf{\bar{x}}, \bar{u}) + \nabla_x \mathbf{f}|_{(\bar{x},\bar{u})} (\mathbf{x} - \mathbf{\bar{x}}) + \frac{\partial \mathbf{f}}{\partial u}|_{(\mathbf{\bar{x}},\bar{u})} (u - \bar{u}). \label{eq:linearization}
	\end{equation}
	
	Neka su $A = \nabla_{\mathbf{x}} \mathbf{f}|_{(\bar{\mathbf{x}},\bar{u})}$, $B = \frac{\partial \mathbf{f}}{\partial u}|_{(\bar{\mathbf{x}},\bar{u})}$, $\Delta \mathbf{x} = \mathbf{x} - \bar{\mathbf{x}}$, i $\Delta u = u - \bar{u}$.
	Prethodna jednačina se može napisati kao:
	\begin{equation}
		\dot{\mathbf{x}} \approx A\Delta \mathbf{x} + B\Delta u + f(\bar{\mathbf{x}}, \bar{u}). \label{eq:linearized_system}
	\end{equation}
	
	Linearizacija se obavlja za različite $\bar{\mathbf{x}}$ čime se dolazi do porodice linearnih modela. Linearni model u okolini tačke  $f(\bar{x} = 0, u = 0) = 0$, može se napisati kao:
	\begin{align}
		\dot{x}(t) &= Ax(t) + Bu(t) \\
		y(t) &= Cx(t) + Du(t)
	\end{align}
	
	
	gde  $x$, $u$, i $y$ označavaju odstupanje od ravnotežnog stanja
	\[
	\mathbf{A} =
	\begin{bmatrix}
		0 & 1.0000 & 0 & 0 & 0 \\
		0 & -0.0174 & 20.7861 & -0.0023 & 57.5344 \\
		0 & 0 & 0 & 1.0000 & 0 \\
		0 & -0.0174 & 63.8319 & -0.0071 & 57.4388 \\
		0 & -232.0252 & 0 & 0 & -755.4250
	\end{bmatrix}
	\]
	
	\[
	\mathbf{B} =
	\begin{bmatrix}
		0 \\
		0 \\
		0 \\
		0 \\
		333.3367
	\end{bmatrix}
	\]
	
	\[
	\mathbf{C} =
	\begin{bmatrix}
		1 & 0 & 0 & 0 & 0 \\
		0 & 0 & 1 & 0 & 0
	\end{bmatrix}
	\]
	
	\[
	\mathbf{D} =
	\begin{bmatrix}
		0 \\
		0
	\end{bmatrix}
	\]
	
	
	\clearpage
	\subsection{Poremećaji u sistemu}
	
	
	
	
	\clearpage
	\subsection{Upravljački signali i skaliranje signala}
	
	
	
	
	\newpage
	
	
	
	\section{Projektovanje sistema upravljanja}
	
	
	\newpage
	\subsection{Generisanje trajektorije}
	
	Ovo poglavlje istražuje kako pronaći optimalan ulaz za dinamički sistem, omogućavajući mu da sledi željenu putanju u prostoru stanja. Ova putanja treba da bude u skladu sa zadanim ograničenjima i optimizuje zadatu veličinu. Iako se različite strategije mogu koristiti za projektovanje putanja, fokus je stavljen na optimalno upravljanje. Ova tehnika pruža jasan okvir za pronalaženje rešenja za probleme podložne ograničenjima i ciljevima optimizacije.
	
	\subsubsection{Ad hoc strategije}
	\label{sec:andhoc}
	U \cite{energy_c} predložen je zakon upravljanja za uspravljanje klatna zasnovanoj na kontroli energije (engl: Energy control): \\
	
	\begin{equation}
		u = \text{\textit{sat}}\left[k_v (E - E_0)\right] \text{\textit{sign}}( \dot{\beta} \cos \beta), 
		\label{eq:en_control}
	\end{equation} \\
	
	
	gde su $\textit{E}$ trenutna energija sistema, 
	$\textit{E}_0$ je željena energija sistema, $\textit{k}_v$ je pojačanje kontrolera, a $\beta$ je ugao penduluma. Prvi pojam definiše amplitudu ulaza. Može se posmatrati kao proporcionalni kontroler, gde je promenljiva razlika u energiji između trenutnog stanja i energije ciljnog stanja, koja je konvencionalno postavljena na nulu. Amplituda je ograničena zbog fizičkih ograničenja aktuatora. Drugi pojam definiše znak upravljačkog ulaza i obezbeđuje da je efekat ulaza dodavanje energije sistemu. Član $\cos \beta$ procenjuje da li je trenutni položaj penduluma iznad ili ispod horizontalnog položaja. Za $\cos \beta = 0$, pendulum je horizontalan, stoga sistem nije kontrolabilan - nema prenosa energije na pendulum. Sa dodatnim članom $\dot{\beta}$, sila se primenjuje protiv smera kretanja penduluma kada je ispod horizontalnog položaja i u istom smeru kada je iznad.\\
	
	Važno je napomenuti da je ovaj zakon upravljanja projektovan specifično za sistem FP, stoga nije vrlo opšte primenljiv. Takođe, ograničen je na situacije gde nivo energije ciljnog stanja nije degenerisan. Iako će kontroler dovesti sistem do stanja sa određenom energijom, ako postoji više konfiguracija sa istom energijom, nije moguće izabrati između njih. Ovo je slučaj kod FP, gde energija sistema ne zavisi od ugla horizontalne ruke $\alpha$. Dakle, ovom tehnikom nema kontrole nad ovom varijablom tokom uspravljanja. \\
	
	Zakon upravljanja definisan u jednačini (\ref{eq:en_control}) zahteva izračunavanje energije sistema. Modifikovana verzija kontrolera predložena je u \cite{energy_c} koja uzima u obzir samo varijable prostora stanja. Ovaj zakon upravljanja poznate je kao 
	Exponentiation of the pendulum position i glasi:
	
	\begin{equation}
		u = \text{\textit{sat}}(k_v |\beta^n|) \text{\textit{sign}}( \dot{\beta} \cos \beta)
	\end{equation} \\
	U ovom slučaju, prvi član je modifikovan, sada uzimajući ugao između trenutnog položaja i vertikale.
	Dodaje se eksponent \textit{n} izražen u izrazu, što povećava amplitudu ulaza kada je pendulum daleko od uspravnog položaja, ali je manja kada je bliže. U \cite{inicijalna} navode da ovaj zakon omogućava glađi prelaz između nelinearnog upravljanja i upravljanja koji balansira klatno u gornjem položaju. Drugi član ostaje nepromenjen. \\
	
	U \cite{ener_shaping} je izveden i testiran zakon upravljanja poznat kao energy shaping. Pomenuti zakon upravljanja je izveden na osnovu inevrznog klatna koji rotira samo oko vertikalne ose i glasi: \\
	
	\begin{equation}
		u = k_1 (\dot\alpha + k_2 \cos(\beta) \dot\beta)	
	\end{equation}
	
	\subsubsection{Optimalno upravljanje}
	
	\newpage
	\subsection{Projektovanje kontrolera}
	
	Nakon generisanja referentnih trajektorije potrebno je projekotvati kontroler čiji je zadatak da u što većoj meri isprati željenu trajektoriju i kontroler koji će održavati sistem u željenom, gornjem položaja. \\
	
	Linearni kontroleri su validni samo u onom regionu gde je izvršena linearizacija sistema, a kako je zadatak upravlanje nelineranim sistemom potrebno je napraviti ansambl kontrolera. \\
	
	U radu \cite{inicijalna} su se bavili projektovanjem LQG kontrolera. Kalman filter je optimalan za procenu stanja u linearnim sistemima, ali kada se primeni na nelinearne sisteme, može doći do netačnih procena stanja zbog odstupanja između stvarne nelinearnosti sistema i linearnog modela koji se koristi u Kalman filtru. Iz ovog razloga, u ovom radu smo se zadržali na korišćenju LQR kontrolera. \\
	
	Cilj algoritma LQR je pronaći vektor pojačanja K za zakon povratne sprege u prostoru stanja
	\begin{equation}
		u = -Kx 
	\end{equation}
	
	koji se primenjuje na kontinualni linearni sistem, definisan u \ref{sec:linearizacija},
	
	\begin{equation}
		\begin{cases}
			\dot{x}(t) = Ax(t) + Bu(t) \\
			y(t) = Cx(t) + Du(t)
		\end{cases}
	\end{equation}
	a minimizuje funkciju troška
	\begin{equation}
		J = \int_{0}^{\infty}  x^T Q_r x  + u^T R_r u
	\end{equation}
	
	Matrice $Q_r$ i $R_r$ određuju relativni značaj koji se pridaje regulaciji stanja i trošku ulaza, i treba ih odabrati uzimajući u obzir ciljeve dizajna sistema.\\
	
	DODATI SLIKU, ref: wikipedija \\
	
	
	Prelazak između kontrolera vrši se u dva različita režima rada: \\
	
	1. Kontrola za podizanje sistema primenjuje se direktno na uređaj. Samo linearni kontroler oko krajnje tačke je aktivan (globalni LQR kontroler). Tranziciji između režima uspravljanja sistema i režima ravnoteže oko uspravnog poožaja se dešava kada se ugao penduluma dovoljno približi nultom položaju ($\beta$ < 18$^\circ$). Ovaj pristup se koristi kada nema referentne putanje, i podizanje sistema se izvodi u zatvorenoj petlji, na primer sa Ad hoc strategijama opisanim u \ref{sec:andhoc}. \\
	
	2. Referentna kontrola se primenjuje na uređaj, koji zauzvrat ima zatvoreni skup linearnih kontrolera koji stabilizuju razliku između izvedene putanje i referentne putanje. Ovo se naziva Gain Scheduling controller.\\
	
	Gain Scheduling controller implementiran u upravljanju FP bira jedan od šest kontrolera koji se koristi u skladu sa trenutnim uglom $\beta$ (slika DODATI iz inicijalne). Važno je napomenuti da su oni raspoređeni simetrično: prvi član Tejlorovog reda je identičan za $\beta = \pm \theta$, pa je i inkrementalni model isti. 
	
	DODATI SLIKU I ZA SEMU iz inicijalne
	
	
	\newpage
	
	\section{Komparativna analiza projektovanih sistema upravljanja}
	
	\subsection{Poređenje odziva sistema}
	\subsubsection{Stabilizacija u gornjem položaju}
	
	\subsubsection{Robustnost na greške u modelovanju}
	\subsubsection{Uticaj šuma}
	
	\subsection{Potiskivanje poremećaja}
	
	
	
	
	\begin{table}[!h]
		\centering
		\begin{tabular}{|llllll|}
			\hline
			\multicolumn{6}{|c|}{Pore\dj{}enje kontrolera}                                                                                                                                                                                                                 \\ \hline
			\multicolumn{1}{|l|}{}              & \multicolumn{1}{l|}{slo"zenost} & \multicolumn{1}{l|}{pra\'cenje ref.} & \multicolumn{1}{l|}{potiskivanje porem.} &  \multicolumn{1}{l|}{multivarijabilnost} & prosek \\ \hline
			\multicolumn{1}{|l|}{$K_{dec}$}     & \multicolumn{1}{l|}{1}          & \multicolumn{1}{l|}{5}                              & \multicolumn{1}{l|}{3}                   &  \multicolumn{1}{l|}{5}    &  2.8   \\ \hline
			\multicolumn{1}{|l|}{$K_{dek0}$}    & \multicolumn{1}{l|}{2}          & \multicolumn{1}{l|}{4}                              & \multicolumn{1}{l|}{2}                   &  \multicolumn{1}{l|}{4}    & 2.4   \\ \hline
			\multicolumn{1}{|l|}{$K_{dek\omega_0}$} & \multicolumn{1}{l|}{3}          & \multicolumn{1}{l|}{3}                              & \multicolumn{1}{l|}{1}                   &  \multicolumn{1}{l|}{3}    & 2   \\ \hline
			\multicolumn{1}{|l|}{$K_{invF}$}  & \multicolumn{1}{l|}{4}          & \multicolumn{1}{l|}{2}                              & \multicolumn{1}{l|}{5}                   &  \multicolumn{1}{l|}{1}    & 2.4   \\ \hline
			\multicolumn{1}{|l|}{$K_{H_{\infty}}$}  & \multicolumn{1}{l|}{5}          & \multicolumn{1}{l|}{1}                              & \multicolumn{1}{l|}{4}                   &  \multicolumn{1}{l|}{2}    & 2.4   \\ \hline
			
		\end{tabular}
		\captionof{table}{}\label{poredjenje_kontrolera}
	\end{table}
	\vspace{1cm}
	
	
	\section{Zaklju"cak}
	
	
	\newpage
	
	\begin{thebibliography}{9}
		\bibitem{inicijalna}
		\emph{Nonlinear control of an inverted pendulum}, António Samuel Ávila Balula, Thesis to obtain the Master of Science Degree in
		Engineering Physics 2016.
		\bibitem{}\emph{Nonlinear pH Control in a CSTR
		} RaynxId A. Wright al Costas Kravais
		\bibitem{beleske}
		\emph{https://automatika.etf.bg.ac.rs/sr/13e054msu}, bele"ske sa predavanja
		
		
		
		\bibitem{energy_c}
		\emph{Swinging up the furuta pendulum and its stabi- ´
			lization via model predictive control.}, P. Seman, B. Rohal’-Ilkiv, M. Juhas, and M. Salaj.  Journal of Electrical Engineering, 64(3):152–158, 2013. ISSN
		13353632. doi: 10.2478/jee-2013-0022.
		\bibitem{ener_shaping}\emph{A normal form for energy shaping: application to the furuta pendulum}
		S. Nair and N. E. Leonard. 
		In Decision and Control, 2002, Proceedings of the 41st IEEE Conference on, volume 1, pages
		516–521. IEEE, 2002
		
	\end{thebibliography}
\end{document}
